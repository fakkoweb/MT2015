
% La bibliografia, da inserirsi solo se ci sono state citazioni.
% In questo caso ricordarsi che bisogna sempre elaborare due volte il file .TEX
% perch� la prima volta viene generata la bibliografia mentre la seconda volta viene inclusa

% NOTA: citare il DOI non � obbligatorio ma MOLTO desiderabile
\iffalse
\begin{thebibliography}{9} % se ci sono meno di 10 citazioni
%\begin{thebibliography}{99} % se ci sono da 10 a 99 citazioni

\bibitem{psisec} % esempio citazione articolo a congresso
I.Enrici, M.Ancilli, A.Lioy, % nomi autori
``A psychological approach to information technology security'', % titolo articolo
HSI-2010: 3rd Int. Conf. on Human System Interactions, % nome del congresso
Rzesz�w (Poland), May 13-15, 2010, % luogo (stato) e data del congresso
pp.\ 459-466, % pagine dell'articolo
\doi{10.1109/HSI.2010.5514528}

\bibitem{tpa} % esempio citazione articolo su rivista
G.Cabiddu, E.Cesena, R.Sassu, D.Vernizzi, G.Ramunno, A.Lioy,  % autori dell'articolo
``Trusted Platform Agent'', % titolo dell'articolo
IEEE Software, % nome della rivista
Vol.\ 28, No.\ 2, % volume e numero della rivista (alcune riviste non ce l'hanno)
March-April 2011, % mese e anno di pubblicazione della rivista
pp.\ 35-41, % pagine dell'articolo
\doi{10.1109/MS.2010.160}


\bibitem{tc} % esempio citazione capitolo di un libro fatto come collezione di contributi da autori diversi
A.Lioy, G.Ramunno, % autori del capitolo
``Trusted Computing'' % titolo del capitolo
nel libro % in the book
``Handbook of Information and Communication Security'' % titolo del libro
a cura di % edited by
P.Stavroulakis, M.Stamp, % nomi dei curatori
Springer, % nome editore
2010, % anno di pubblicazione
pp.\ 697-717, % pagine del capitolo
\doi{10.1007/978-3-642-04117-4_32}

\bibitem{openssl} % esempio citazione pagina web di un progetto
The OpenSSL project, % nome del progetto
\url{http://www.openssl.org/} % URI della pagina web

\bibitem{tls12} % esempio citazione RFC
T.Dierks, E.Rescorla,
``The Transport Layer Security (TLS) Protocol Version 1.2'',
\rfc{5246}, August 2008

\bibitem{seceng} % esempio citazione libro
Ross J. Anderson,
``Security engineering'',
Wiley, 2008,
ISBN: 978-0-470-06852-6


\end{thebibliography}
\fi
\clearpage
1 - zero point: https://www.youtube.com/watch?v=DsXEUPS2uss \\ 
2 - star wars trailer: https://www.facebook.com/StarWars/videos/1030579940326940/ \\
3 - PRESENCE AS THE DEFINING FACTORIN A VR APPLICATION -- \\ https://smartech.gatech.edu/bitstream/handle/1853/3584/94-06.pdf \\ 
4 - Google visual translate, formerly World Lens -- \\ https://play.google.com/store/apps/details?id=com.google.android.apps.translate \\
5 - IKEA virtual forniture -- https://www.youtube.com/watch?v=vDNzTasuYEw \\
6 - Augmented Reality: A class of displays on the reality-virtuality continuum -- \\ http://web.cs.wpi.edu/$\sim$gogo/hive/papers/Milgram\_Takemura\_SPIE\_1994.pdf \\
7 - Optical Versus Video See-Through Head-Mounted Displays in Medical Visualization \\
8 - Alfano, Patricia L. and George F. Michel. (1990). Restricting the field of view: perceptual and performance effects. Perceptual and Motor Skills, 70 (1), 35-45 \\
9 - GPU Accelerated Realtime Stereo for Augmented Reality \\
10 - working distance zero plane \\
11 - infinite zero plane \\
12 - Binocular Disparity and the Perception of Depth, Ning Quian, 1997 -- \\ http://brahms.cpmc.columbia.edu/publications/stereo-review.pdf \\
13 - già vistooo \\
14 - The Modelling of Stereoscopic 3D Scene Acquisition, paragraph 1.1 \\ 
15 - The Camera Convergence Problem Revisited - Robert S. Allison \\ 
16 - Dynamic Eye Convergence for Head-mounted Displays Improves User Performance in Virtual Environments \\
17 - Creating correct stereo pairs from any raytracer, Paul Burke, 2001 \\
18 - Offaxis frustums: What are they and what are they good for?, Paul Burke, 2004 presentation \\
19 - AR rift page \\
20 - Book \\
21 - Digital stereo photography tricks and effects, Donald E. Simanek, \\ https://www.lhup.edu/$\sim$dsimanek/3d/stereo/tricks.htm \\
22 - Example of Stereo Calibration using OpenCV, Jay Rambhia, 30 Mar 2013 \\
23 - Stereo rectification of calibrated image pairs based on geometric transformation \\
24 - Calculating Stereo Pairs, Paul Burke, 1999 \\ 
25 - Automatic Keystone Correction for Camera-assisted Presentation Interfaces, 2000, R. Sukthankar, R. Stockton, M. Mullin \\
26 - Create side-by-side stereo pairs in the Unity game engine, Paul Burke, 2008 \\
27 - Optical Versus Video See-Through Head-Mounted Displays in Medical Visualization, 3.2.3 Adaptation \\
28 - www.sub2r.com \\
29 - www.ptgrey.com \\
** 30 - Presence and Discernability in Conventional and Non-Photorealistic Immersive Augmented Reality, W. Steptoe, S. Julier, A. Steed, 2015 \\
31 - book \\
32 - Stereographic Projection Techniques for Geologists and Civil Engineers \\
33 - Lantz, E. A survey of large-scale immersive displays. Proceedings of the 2007 workshop on Emerging displays technologies: images and beyond: the future of displays and interaction. ACM International Conference Proceeding Series; Vol. 252, 2007, ISBN:978-1-59593-669-1 \\
34 - Omni-directional Stereoscopic Fisheye Images for Immersive Hemispherical Dome Environments, Paul Burke,  \\
35 - Salamon, D. 2005. Transformations and Projections in Computer Graphics. Springer London, pp 145-220. DOI: 10.1007/978-1-84628-620-9. \\
36 - The magic of computer graphics, Noriko Kurachi, CRC Press, 2011, ch5: "image based rendering" \\
37 - Vrui webpage \\
38 - Example post "Correcting Fisheye Distortion Programmatically" -- \\ http://stackoverflow.com/questions/2477774/correcting-fisheye-distortion-programmatically \\
39 - PRECISE METHOD OF FISHEYE LENS CALIBRATION - M. Kedzierski, A. Fryskowska \\
40 - Calib3d implementation in OpenCV for fisheye model - \\ http://docs.opencv.org/master/db/d58/group\_\_calib3d\_\_fisheye.html\#gsc.tab=0 \\
41 - OcamCalib for Matlab -- https://sites.google.com/site/scarabotix/ocamcalib-toolbox \\
42 - arUco with Ogre example - https://www.youtube.com/watch?v=CzD48UkGsK8