% La bibliografia, da inserirsi solo se ci sono state citazioni.
% In questo caso ricordarsi che bisogna sempre elaborare due volte il file .TEX
% perch� la prima volta viene generata la bibliografia mentre la seconda volta viene inclusa

% NOTA: citare il DOI non � obbligatorio ma MOLTO desiderabile

%\begin{thebibliography}{9} % se ci sono meno di 10 citazioni
\begin{thebibliography}{99} % se ci sono da 10 a 99 citazioni

\bibitem{immersivity_vr}
D. Marini, R. Folgieri, D. Gadia, A. Rizzi,
``Virtual reality as a communication process", Virtual Reality 16, no. 3 (2012): 233-241.


\bibitem{linkzeropoint}
Youtube - ``Zero Point", VR 360 film, \\ \href{https://www.youtube.com/watch?v=DsXEUPS2uss}{https://www.youtube.com/watch?v=DsXEUPS2uss}

\bibitem{link_starwars_trailer}
Facebook - ``Star Wars: The Force Awakens", Immersive 360 Teaser Trailer, \\ \href{https://www.facebook.com/StarWars/videos/1030579940326940/}{https://www.facebook.com/StarWars/videos/1030579940326940/}


\bibitem{immersivity_film}
A. D’Aloia,
``Film in Depth. Water and Immersivity in the Contemporary Film Experience." Acta Universitatis Sapientiae, Film and Media Studies 05 (2012): 87-106.

\bibitem{immersivity_3D}
M. Kozhevnikov, R. P. Dhond,
``Understanding immersivity: image generation and transformation processes in 3D immersive environments." Frontiers in psychology 3 (2012).

\bibitem{book_cv}
R. Hartley and A. Zisserman,
``Multiple View Geometry in Computer Vision", Second Edition, Cambridge University Press, March 2004,
ISBN: 0521540518

\bibitem{ar_intro}
 R. T. Azuma,
``A survey of augmented reality" Presence 6, no. 4 (1997): 355-385.

\bibitem{telepresence_intro}
G. H. Ballantyne,
``Robotic surgery, telerobotic surgery, telepresence, and telementoring", Surgical Endoscopy and Other Interventional Techniques 16, no. 10 (2002): 1389-1402.


\bibitem{vr_presence}
Larry F.~Hodges, Barbara O.~Rothbaum, R.~Kooperα,
D.~Opdyke, T.~Meyer, Johannes J. de Graaff,
James S. Williford, Max M. North,
``Presence as the defining factor in a VR application",
GVU Technical Report, GIT-GVU-94-06,
Georgia Institute of Technology (USA), 1994


\bibitem{link_google_translate_AR}
Google Play App Store - ``Google Translate", formerly World Lens, \\ \href{https://play.google.com/store/apps/details?id=com.google.android.apps.translate}{https://play.google.com/store/apps/details?id=com.google.android.apps.translate}

\bibitem{link_IKEA_AR}
Youtube - ``IKEA AR Catalogue", commercial, \\ \href{https://www.youtube.com/watch?v=vDNzTasuYEw}{https://www.youtube.com/watch?v=vDNzTasuYEw}

\bibitem{milgram_continuum}
P. Milgram, H. Takemura, A. Utsumi, F. Kishino,
``Augmented Reality: A class of displays on the reality-virtuality continuum",
Proc. SPIE 2351, Telemanipulator and Telepresence Technologies, 282 (December 21, 1995), \doi{10.1117/12.197321}

\bibitem{tracking_AR}
R. Azuma,
``Tracking requirements for augmented reality", Communications of the ACM 36, no. 7 (1993): 50-51.




\bibitem{optical_vs_video_st}
J. P. Rolland, H. Fucks,
``Optical Versus Video See-Through Head-Mounted Displays in Medical Visualization",
Massachusetts Institute of Technology, June 2000, Vol. 9, No. 3, Pages 287-309,
\doi{10.1162/105474600566808}

\bibitem{virtual_sickness}
J. J. W. Lin, H. B. Duh, D. E. Parker, H. Abi-Rached, T. Furness,
``Effects of field of view on presence, enjoyment, memory, and simulator sickness in a virtual environment", in Virtual Reality, 2002. Proceedings. IEEE, pp. 164-171. IEEE, 2002.

\bibitem{oculus_rift}
Oculus Rift official homepage, 2015 \\ \href{https://www.oculus.com/en-us/rift/}{www.oculus.com/en-us/rift/}

\bibitem{google_glass}
Google Glass official homepage, 2015 \\ \href{https://www.google.com/glass/start/
}{www.google.com/glass/start/}

\bibitem{microsoft_hololens}
Microsoft Hololens project official homepage, 2015 \\ \href{https://www.microsoft.com/microsoft-hololens/en-us}{www.microsoft.com/microsoft-hololens/en-us}

\bibitem{restricting_FOV}
P. L. Alfano, G. F. Michel
``Restricting the field of view: perceptual and performance effects",
DePaul University, 1990, Perceptual and Motor Skills: Volume 70, Issue , pp. 35-45,
\doi{10.2466/pms.1990.70.1.35}

\bibitem{GPU_accel_stereo_AR}
M. Sizintsev, S. Kuthirummal, S. Samarasekera, R. Kumar, H. S. Sawhney, A. Chaudhry,
``GPU accelerated realtime stereo for augmented reality", 2010, In Proceedings Intl. Symp. 3D Data Processing, Visualization and Transmission (3DPVT)

\bibitem{stereo_rules}
F. Zilly, J. Kluger, P. Kauff,
``Production rules for stereo acquisition", Proceedings of the IEEE 99, no. 4 (2011): 590-606,
\doi{10.1109/JPROC.2010.2095810}

\bibitem{light_field_mapping}
WC. Chen, JY. Bouguet, M. H. Chu, R. Grzeszczuk,
``Light field mapping: efficient representation and hardware rendering of surface light fields", in ACM Transactions on Graphics (TOG), vol. 21, no. 3, pp. 447-456. ACM, 2002.

\bibitem{magic_leap}
Magic Leap project homepage, 2015 \\ \href{http://www.magicleap.com/}{magicleap.com}

\bibitem{disparity_depth}
N. Quian,
``Binocular Disparity and the Perception of Depth",
Neuron, Elsevier, Volume 18, Issue 3, p359–368, March 1997,
\doi{10.1016/S0896-6273(00)81238-6}



\bibitem{stereoscopic_3D_acquisition}
M. Hasmanda, K. Riha,
``The Modelling of Stereoscopic 3D Scene Acquisition",
Brno University of Technology, April 2012, cited from paragraph 1.1

\bibitem{correct_stereo_pairs}
P. Burke,
``Creating correct stereo pairs from any raytracer",
SPIE Three Dimensional Imaging and Remote Sensing Imaging, Vol 902, pp 85, 2001

\bibitem{camera_convergence}
R. S. Allison,
``The Camera Convergence Problem Revisited",
Department of Computer Science and Centre for Vision Research, York
University, 2004,
\doi{10.1.1.145.9490}

\bibitem{dynamic_virtual_eye_convergence}
A. Sherstyuk, A. Dey, C. Sandor,
``Dynamic eye convergence for head-mounted displays improves user performance in virtual environments",
Proceedings of the ACM SIGGRAPH Symposium on Interactive 3D Graphics and Games. ACM, 2012


\bibitem{offaxis_frustrums}
P. Burke,
``Offaxis frustums: What are they and what are they good for?"
Centre For Astrophysics and Supercomputing, Swinburne University, HET409, September 2004

\bibitem{link_toein_diffused}
O. Kreylos,
``Good Stereo vs Bad Stereo", a brief story of most diffused stereo technique, 2012 \\ \href{http://doc-ok.org/?tag=lens-shift}{http://doc-ok.org/?tag=lens-shift}

\bibitem{link_arrift}
W. Steptoe,
``AR-Rift", Augmented reality project for NPR perception, 2014 \\ \href{http://willsteptoe.com/post/66968953089/ar-rift-part-1}{http://willsteptoe.com/post/66968953089/ar-rift-part-1}

\bibitem{link_stereo_tricks}
D. E. Simanek,
``Digital stereo photography tricks and effects" \\ \href{https://www.lhup.edu/~dsimanek/3d/stereo/tricks.htm}{https://www.lhup.edu/~dsimanek/3d/stereo/tricks.htm}
May, 2010

\bibitem{link_stereo_calib_example}
J. Rambhia,
``Stereo Calibration", an example of stereo calibration using OpenCV \\ \href{http://www.jayrambhia.com/blog/stereo-calibration/}{http://www.jayrambhia.com/blog/stereo-calibration/}
March, 2013

\bibitem{stereo_rectify_parallelise}
H. Su, B. W. He,
``Stereo rectification of calibrated image pairs based on geometric transformation",
I.J. Modern Education and Computer Science (MECS), 2011, 4, 17-24

\bibitem{link_oculus_limits}
O. Kreylos
``A Closer Look at the Oculus Rift", March 2014, \\ \href{http://doc-ok.org/?p=756}{http://doc-ok.org/?p=756}

\bibitem{oculus_prediction}
LaValle, Steven M., Anna Yershova, Max Katsev, and Michael Antonov. ``Head tracking for the Oculus Rift." In Robotics and Automation (ICRA), 2014 IEEE International Conference on, pp. 187-194. IEEE, 2014.

\bibitem{latency_sinequanon}
M. Abrash,
``Latency: the sine qua non of AR and VR", December 2012 \\ \href{http://blogs.valvesoftware.com/abrash/latency-the-sine-qua-non-of-ar-and-vr}{http://blogs.valvesoftware.com/abrash/latency-the-sine-qua-non-of-ar-and-vr}

\bibitem{karmack_mitigation}
J. Carmack, ``Latency mitigation strategies", Feb. 2013 \\ Backup link: \href{https://www.twentymilliseconds.com/post/latency-mitigation-strategies/}{https://www.twentymilliseconds.com/post/latency-mitigation-strategies/}






\bibitem{stereo_pair_cameras}
P. Burke,
``Calculating Stereo Pairs", 1999 \\ \href{http://paulbourke.net/stereographics/stereorender/}{http://paulbourke.net/stereographics/stereorender/}

\bibitem{keystone_correction}
R. Sukthankar, R. Stockton, M. Mullin,
``Automatic Keystone Correction for Camera-assisted Presentation Interfaces",
Proceedings of International Conference on Multimedia Interfaces,
October, 2000

\bibitem{stereo_pairs_game}
P. Burke,
``Create side-by-side stereo pairs in the Unity game engine", 2008 \\ \href{http://paulbourke.net/stereographics/Unitystereo/}{http://paulbourke.net/stereographics/Unitystereo/}


\bibitem{sub2r}
SUB2R - High speed cameras for computer vision \\ \href{http://www.sub2r.com}{http://www.sub2r.com}

\bibitem{ptgrey}
PTGREY - High speed cameras for computer vision \\ \href{http://www.ptgrey.com}{http://www.ptgrey.com}

\bibitem{ar_rift}
W. Steptoe, S. Julier, A. Steed,
``Presence and Discernability in Conventional and Non-Photorealistic Immersive Augmented Reality",
Mixed and Augmented Reality (ISMAR), 2014 IEEE International Symposium on. IEEE, 2015,
\doi{10.1109/ISMAR.2014.6948430}




\bibitem{fisheye_lens}
D. Brooks,
`` Lenses and lens accessories: a photographer's guide", 1982, p. 29. \\ISBN: 9780930764340.

\bibitem{book_stereographic_projection}
R. J. Lisle, P. R. Leyshon,
``Stereographic Projection Techniques for Geologists and Civil Engineers",
Cambridge University Press, 2004
\doi{10.1017/CBO9781139171366}

\bibitem{immersive_displays}
E. Lantz, 
``A survey of large-scale immersive displays", Proceedings of the 2007 workshop on Emerging displays technologies: images and beyond: the future of displays and interaction, ACM International Conference Proceeding Series; Vol. 252, 2007,
\doi{10.1145/1278240.1278241}

\bibitem{omni_fisheye}
P. Burke,
``Omni-directional Stereoscopic Fisheye Images for Immersive Hemispherical Dome Environments",
WASP, University of Western Australia, Computer Games and Allied Technology,136-143, 2009

\bibitem{cg_projections}
D. Salamon, 
``Transformations and Projections in Computer Graphics", Springer London, pp 145-220, 2005,
\doi{10.1007/978-1-84628-620-9}


\bibitem{link_dissecting_camera_matrix}
K. Simek, ``Dissecting the camera matrix", June 2013\\ \href{http://ksimek.github.io/2013/06/03/calibrated\_cameras\_in\_opengl/}{http://ksimek.github.io/2013/06/03/calibrated\_cameras\_in\_opengl/}


\bibitem{cg_rendering}
N. Kurachi, 
``The magic of computer graphics", CRC Press, cited from Chapter 5: "image based rendering", 2011,\\
Print ISBN: 978-1-56881-577-0\\
eBook ISBN: 978-1-4398-7357-1

\bibitem{link_vrui}
VRUI - Vrui Toolkit Official Homepage \\ \href{http://idav.ucdavis.edu/~okreylos/ResDev/Vrui/}{http://idav.ucdavis.edu/~okreylos/ResDev/Vrui/}

\bibitem{link_fisheye_undistortion}
Stackoverflow - ``Correcting Fisheye Distortion Programmatically", an example result of fish-eye undistortion and its problems \\ \href{http://stackoverflow.com/questions/2477774/correcting-fisheye-distortion-programmatically}{http://stackoverflow.com/questions/2477774/correcting-fisheye-distortion-programmatically}

\bibitem{precise_fisheye_calib}
M. Kedzierski, P. Walczykowski, R. Kaczynski,
``Precise calibration of fisheye lens camera system and projection model", 2006

\bibitem{link_calib3d_opencv}
OpenCV - Calib3d implementation in OpenCV for fisheye model \\ \href{http://docs.opencv.org/master/db/d58/group_\_calib3d_\_fisheye.html}{http://docs.opencv.org/master/db/d58/group\_\_calib3d\_\_fisheye.html}


\bibitem{link_matlab_ocamcalib}
OcamCalib - Camera calibration tool for Matlab \\ \href{https://sites.google.com/site/scarabotix/ocamcalib-toolbox}{https://sites.google.com/site/scarabotix/ocamcalib-toolbox}

\bibitem{link_aruco_ogre}
arUco - Example with Ogre3D \\ \href{https://www.youtube.com/watch?v=CzD48UkGsK8}{https://www.youtube.com/watch?v=CzD48UkGsK8}

\bibitem{link_optimized_undistort_opencv}
openCV - How to improve OpenCV performance on lens undistortion from a video feed \\ \href{http://blog.nishihara.me/opencv/2015/09/03/how-to-improve-opencv-performance-on-lens-undistortion-from-a-video-feed/}{http://blog.nishihara.me/opencv/2015/09/03/how-to-improve-opencv-performance-on-lens-undistortion-from-a-video-feed/}

\bibitem{leap_motion}
Leap Motion official homepage, 2015 \\ \href{https://www.leapmotion.com/}{www.leapmotion.com}

\iffalse
\fi


\end{thebibliography}
