\section{Taxonomy}

A VR/AR setup must deal with different disciplines and integrate them fluently. We will now analyse the most important aspects the work described in the next chapter takes from, belonging to the following fields of expertise:
\begin{itemize}
\item
computer vision, which covers the techniques for displaying/enhancing images and extrapolating numerical or symbolic information about the real world. Optics and imaging sensors in general are modelled and therefore used to extract that data;
\item
stereoscopy, a collection of techniques addressing the specific problem of creating or enhancing the illusion of depth in images, exploiting principles in optics and human perception. Most methods involve two separate images to achieve this effect. A combination of such techniques and computer vision models opens the door to computer stereo vision, where more than one image is used to extract data, thus requiring less known parameters from the scene;
\item
virtual reality, a discipline that implies mathematical models, actuators and sensors to replicate real world experiences in form of entirely computer-simulated environments and give them interaction capabilities. High relevance is given to human psychology and perception limits in order to recreate a lifelike experience. Images are in this case entirely synthetic: models and image enhancements implied have lot in common with computer vision, although the latter focuses on analysis. We find the most common expression of their combination in what is called augmented reality.
\end{itemize}

\section{Augmented reality vs Augmented Virtuality}


\subsection{Optical vs Video See-through} %limits and advantages

\section{Immersivity}

\subsection{Resolution}
\subsection{Frame Rate}
\subsection{Refresh Rate}
\subsection{Latency}

