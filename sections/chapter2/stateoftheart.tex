A VR/AR setup must deal with different disciplines and integrate them fluently. We will now analyse the most important aspects the work described in the next chapter takes from, belonging to the following fields of expertise:
\begin{itemize}
\item
computer vision, which covers the techniques for displaying/enhancing images and extrapolating numerical or symbolic information about the real world. Optics and imaging sensors in general are modeled and therefore used to produce that data;
\item
stereoscopy, a collection of techniques addressing the specific problem of creating or enhancing the illusion of depth in images, exploiting principles in optics and human perception. Most methods involve two separate images to achieve this effect. A combination of such techniques and computer vision models opens the door to computer stereo vision, where more than one image is used to retrieve information on a scene;
\item
virtual reality, a discipline that uses mathematical models, actuators and sensors to replicate real world experiences in form of entirely computer-simulated environments and give them interaction capabilities. High relevance is given to human psychology and perception limits in order to recreate a lifelike experience. Images are in this case entirely synthetic: models and image enhancements implied have lot in common with computer vision, although the latter focuses on analysis. A combination
\item
un esame a scelta tra Chimica ed Informatica.
\end{itemize}

virtual reality: models mathematically the real world and aims to reproduce that feel in human users. It comprehends also techniques for image enhancement, but in this case images are entirely generated by the model. Also in this case a model is used for a camera observing a scene, even though fictional.

\section{}
A VR/AR setup/application must deal with different aspects. Three are the main fields of expertise:
computer vision: describes the techniques for displaying/enhancing images and extrapolating information by means of imaging sensors as cameras according to different models. Stereo vision addresses a specific set of  problems regarding the use of two cameras in order to achieve more information and the stereoscopic effect similar to the human vision.
virtual reality: models mathematically the real world and aims to reproduce that feel in human users. It comprehends also techniques for image enhancement, but in this case images are entirely generated by the model. Also in this case a model is used for a camera observing a scene, even though fictional.
