% per commentare una riga mettere % al suo inizio
% per s-commentare una riga (ossia attivarla) togliere il % al suo inizio
%
\documentclass[pdfa% formato PDF/A, obbligatorio per l'archiviazione delle tesi di Polito
,cucitura%lascia margine per la rilegatura
%,twoside% per stampa fronte-retro (fortemente consigliato per tesi voluminose, opzionale per le altre)
%,12pt% font pi� grande (12pt) rispetto a quello normalmente usato (11pt)
]{toptesi}
%
% Commentare le righe seguenti se NON si � specificata l'opzione "pdfa"
\hypersetup{%
    pdfpagemode={UseOutlines},
    bookmarksopen,
    pdfstartview={FitH},
    colorlinks,
    linkcolor={blue},
    citecolor={red},
    urlcolor={blue}
  }
% \documentclass[11pt,twoside,oldstyle,autoretitolo,classica,greek]{toptesi}
% \usepackage[or]{teubner}
%%%%%%%%%%%%%%%%%%%%%%%%%%%%%%%%%%%%%%%%%%%%%%%%%%%%
%
% Esempio di composizione di tesi di laurea.
%
% Questo esempio e' stato preparato inizialmente 13-marzo-1989
% e poi e' stato modificato via via che TOPtesi andava
% arricchendosi di altre possibilita'.
%
% Nel seguito laurea "quinquennale" sta anche per "specialistica" o "magistrale"

% Cambiare encoding a piacere; oppure non caricare nessun encoding se si usano
% solo caratteri a 7 bit (ASCII) nei file d'entrata.
%
\usepackage[latin1]{inputenc}% IMPORTANTE! usare codifica ISO-8859-1 per le lettere accentate


%
%\chapterbib %solo per vedere che cosa succede; e' preferibile comporre una sola bibliografia
%\AdvisorName{Supervisors}
%\newtheorem{osservazione}{Osservazione}% Standard LaTeX

%\usepackage[a-1b]{pdfx}
%\hypersetup{%
%    pdfpagemode={UseOutlines},
%    bookmarksopen,
%    pdfstartview={FitH},
%    colorlinks,
%    linkcolor={blue},
%    citecolor={green},
%    urlcolor={blue}
%  }

%
% per numerare e far comparire nell'indice anche le sezioni di quarto livello
% SCONSIGLIATO! da usarsi solo in caso di estrema necessit�
%\setcounter{secnumdepth}{4}% section-numbering-depth
%\setcounter{tocdepth}{4}% TOC-numbering-depth (TOC=Table-Of-Content)

%\setbindingcorrection{3mm}

% Set default language (if not defined, then is "italiano"
% To switch to other language during writing, use \selectlanguage{language}
\english


%%%%%%%%%%%%%%%%%%%%%%%%%%%%%%%%%%%%%%%%%
%%%%%%% Change the strings if you want a title page and a copyright page
%%%%%%% in another language
%%%%%%% Comment just the \iflanguage statement and the closing line of the language test
%%%%%%% if you want to make a global change instead of a conditional one.
%%%%%%% Comment the following indented lines if you don't care about the title page
%%%%%%% in English
\iflanguage{english}{%
	%\retrofrontespizio{This work is subject to the Creative Commons Licence}
	\DottoratoIn{PhD Course in\space}
	\CorsoDiLaureaIn{Master in\space}
	\NomeMonografia{Bachelor Degree Final Work}
	\TesiDiLaurea{Master Thesis}
	\NomeDissertazione{PhD Dissertation}
	\InName{in}
	\CandidateName{Candidates}% or Candidate
	\AdvisorName{Supervisors}% or Supervisor
	\TutorName{Tutor}
	\NomeTutoreAziendale{Internship Tutor}
	\CycleName{cycle}
	\NomePrimoTomo{First volume}
	\NomeSecondoTomo{Second Volume}
	\NomeTerzoTomo{Third Volume}
	\NomeQuartoTomo{Fourth Volume}
}{}
%%%%%%%%%%%%%%%%%%%%%%%%%%%%%%%%%%%%%%%%%



\ateneo{Politecnico di Torino}

%%% scegliere la propria facoltà (solo PRIMA dell'AA 2012-2013)
%
%\facolta[III]{Ingegneria dell'Informazione}
%\facolta[IV]{Organizzazione d'Impresa\\e Ingegneria Gestionale}
%\Materia{Remote sensing}% uso sconsigliato
\FacoltaDi		% no facoltà di voice
\facolta			% no facoltà name

%\monografia{Gestione informatizzata di un magazzino ricambi}% per la laurea triennale
\titolo{The challenges of immersive stereo augmented reality in wide field of view video head mounted displays}% per la laurea quinquennale e il dottorato
%\sottotitolo{Metodo dei satelliti medicei}% NON obbligatorio, per la laurea quinquennale e il dottorato

%%% scegliere il proprio corso
%
%\corsodilaurea{Ingegneria dell'Organizzazione d'Impresa}% per la laurea di primo e secondo livello
%\corsodilaurea{Ingegneria Logistica e della Produzione}% per la laurea di primo e secondo livello
%\corsodilaurea{Ingegneria Gestionale}% per la laurea di primo e secondo livello
\corsodilaurea{Computer Engineering}% per la laurea di primo e secondo livello
%\corsodidottorato{Meccanica}% per il dottorato

\candidato{Dario \textsc{Facchini}}% per tutti i percorsi
%\secondocandidato{Evangelista \textsc{Torricelli}}% per la laurea magistrale solamente
%\direttore{prof. Albert Einstein}% per il dottorato
%\coordinatore{prof. Albert Einstein}% per il dottorato
\relatore{prof.\ Andrea Bottino}% per la laurea e il dottorato
%\secondorelatore{dipl.~ing.~Werner von Braun}% per la laurea magistrale
%\terzorelatore{{\tabular{@{}l}dott.\ Neil Armstrong\\prof. Maria Rossi\endtabular}}% per la laurea magistrale
%\tutore{ing.~Karl Von Braun}% per il dottorato
%\tutoreaziendale{dott.\ ing.\ Giovanni Giacosa} % solo per la laurea di secondo livello con tesi svolta in azienda
%\NomeTutoreAziendale{Supervisore aziendale\\Centro Ricerche FIAT}
%\sedutadilaurea{Agosto 1615}% per la laurea quinquennale
%\esamedidottorato{Novembre 1610}% per il dottorato
\sedutadilaurea{\textsc{Dicembre} 2015}% per la laurea triennale
%\sedutadilaurea{\textsc{Anno~accademico} 1615-1616}% per la laurea magistrale
%\annoaccademico{1615-1616}% solo con l'opzione classica
%\annoaccademico{2006-2007}% idem
%\ciclodidottorato{XV}% per il dottorato
\logosede{sections/frontespizio/logopolito}

\input{commands/commands.tex}

\begin{document}



\errorcontextlines=9

\expandafter\ifx\csname StileTrieste\endcsname\relax
    \frontespizio
\else
    \paginavuota
    \begin{dedica}
        A mio padre

        \textdagger\ A mio nonno Pino
    \end{dedica}
    \tomo
\fi

% COMMENT OUT TO REMOVE THE FOLLOWING SECTIONS

% prints summary of the content
\sommario

In this work we investigate the challenges of immersive AR by means of a stereoscopic, wide field-of-view (FOV) head-mounted display (HMD) and subsequent realization of see-through with a stereo camera rig. Nowadays Virtual Reality (VR) brings virtual world closer to realistic experience while modern computer vision offers better digital understanding of reality. Stereoscopy is a veteran discipline of which both make use but not much has been done to fill the gap between the two. By borrowing both the perceptual aspects of immersive virtual reality and image analysis and elaboration from computer vision, we attempt to successfully blend synthetic and real images that can possibly interact and bond with Augmented Reality (AR). We present a set of strategies which proved to help healing the encountered latency, perception and geometrical discrepancies and tested them on Oculus Rift DK2 and a custom stereo camera rig. As a result, a final representation model and architecture was developed to decouple vision, stereoscopy and computer graphics implementation aspects and a configurable solution was provided to coexist in one application.

% prints acknowledgements and special thanks
\ringraziamenti

Opzionali, solo nel caso si sia ricevuto un aiuto speciale e particolarmente rilevante.

%% inserire sempre nella tesi per la laurea di I livello, perché il nome dei tutori non è indicato sul frontespizio.
%Il lavoro descritto in questa monografia è stato svolto sotto la supervisione
%del Prof. Antonio Lioy (tutore accademico)% inserire sempre il nome del tutore accademico
% e dell'Ing. Mario Rossi (tutore aziendale)% inserire solo se la monografia è relativa ad un tirocinio.
%.



%\tablespagetrue % normalmente questa riga non serve ed e' commentata
%\figurespagetrue % normalmente questa riga non serve ed e' commentata

% prints the thesis index section
\indici

% start chapter enumeration
\mainmatter

\chapter{Introduction}
\input{latex.tex}
%\section{Immersivity and vision}
The first thing that probably comes to mind when talking about 3D and stereoscopy is games and movies. These two happen to be the most popular and remunerative mass entertainment media nowadays, apparently carrying forward very different imagery philosophies: cinema was born for telling carefully prefabricated stories, meanwhile a game had interaction at its core and one’s actions would make his own stories. They have actually very much in common, especially regarding how images are perceived or must be perceived by an human observer. 

Needless to say, they inevitably crossed their paths in a war where each one wanted to be a little more of the other. Games borrowed Hollywood-like storytelling while movies experimented with 3D environments. Recently film makers are exploring the possibilities of full 360 degree imagery (Figures \ref{fig:zeropoint_demo} and \ref{fig:starwars_trailer}), while gaming industry is pushing to make interactive virtual reality a solid foundation. Both are today focus on the battle of "immersivity", a term referring to how synthetic content generates mental information in individuals so that it is experienced as close as possible to real \cite{immersivity_vr}.

%\captionsetup{margin=2cm}
%\captionsetup{justification=centering}
\begin{figure}
\centering
\begin{subfigure}{0.49\textwidth}
\centering
\includegraphics[scale=0.15]{pictures/zeropoint_demo.png}
\caption{Zero Point VR 360 film on Oculus Rift DK2 (source: \href{https://www.youtube.com/watch?v=DsXEUPS2uss}{youtube.com}) \cite{linkzeropoint}}
\label{fig:zeropoint_demo}
\end{subfigure}
\begin{subfigure}{0.49\textwidth}
\centering
\includegraphics[scale=0.15]{pictures/starwars_trailer.png}
\caption{Star Wars: The Force Awakens Immersive 360 Trailer - different viewpoint screenshots (source: \href{https://www.facebook.com/StarWars/videos/1030579940326940/
}{facebook.com}) \cite{link_starwars_trailer}}
\label{fig:starwars_trailer}
\end{subfigure}
\caption{Two examples of modern union of VR and cinema-like experience}
\label{fig:VR_cinema_examples}
\end{figure}
Current research has been tackling the challenge of what is really immersive and in what ways user experience can be enhanced by means of methods proposed by the previously mentioned doctrines, which stretch from photography and optics \cite{immersivity_film} to computer graphics and parallel computation \cite{immersivity_3D}. At the same time, the more veteran branch of computer vision has been researching how computationally relevant features can be extracted from real world imagery and put use to it by giving machines comprehension of what is happening around them \cite{book_cv}. Such information can finally support also humans in various tasks where imaging devices are involved, but still computer vision does not focus on better ways to deliver it to end users. Recent marriage of computer vision and virtual reality gave birth to Augmented Reality (AR), a step forward on this matter: target applications would provide more intuitive, interactive and integrated human-machine interfaces, coherently with the world observed \cite{ar_intro}.

Less explored is the path where all mentioned features find a place in applications where contact with real world is severed or hijacked and how visual coherence can be enhanced the other way around; the field of studies where human perception is involved can extend its applications from more immersive AR tele-presence \cite{telepresence_intro} to the possibility to actual artificially augment human perception capabilities. In this work we investigate the challenges of immersive AR by means of a stereoscopic, wide field-of-view (referred from now on as "FOV") head-mounted display (referred as "HMD", head mounted display) and subsequent realization of see-through with cameras. A custom built stereo camera rig used to achieve stereoscopic images showing in the headset what is directly in front of the user, proposing a way to eventually alter them and/or extract scene information with classic computer vision techniques. Our implementation proposes a uniform way The scenario introduces many real-world effects including latency, optical artefacts, depth perception and other and mismatches in real and virtual content: our implementation proposes a way to uniformly face those problems, by proposing a processing pipeline oriented to be platform independent and easy to experiment with for both human perception and computer vision algorithms.

In such attempt, we will traverse common aspects of top-notch techniques for human-computer interaction and perception, belonging to the following fields of expertise:
\begin{itemize}
\item computer vision, which covers the techniques for displaying/enhancing images and extrapolating numerical or symbolic information about the real world. Optics and imaging sensors in general are modelled and therefore used to extract that data, which can range from bare geometrical information (such as distances, position and orientation of a camera or known objects in the scene) to more complex pattern recognition;
\item stereoscopy, a collection of techniques addressing the specific problem of creating or enhancing the illusion of depth in images, exploiting principles in optics and human perception. Most methods involve two separate images to achieve this effect. A combination of such techniques and computer vision models opens the door to computer stereo vision, where more than one image is used to extract data, thus requiring less known parameters from the scene;
\item virtual reality, a discipline that implies mathematical models, actuators and sensors to replicate real world experiences in real time in form of entirely computer-simulated environments and give them interaction capabilities. High relevance is given to human psychology and perception limits in order to recreate a lifelike experience. Images are in this case entirely synthetic: models and image enhancements implied have lot in common with computer vision, although the latter focuses on analysis. We find the most common expression of their combination in what is called augmented reality.
\end{itemize}

In the specific we will experiment with Augmented Reality/Virtuality, with a big focus on high visually immersive applications through the use of a virtual reality headset. A custom built stereo camera rig is used to achieve stereoscopic images showing in the headset what is directly in front of the user, proposing a way to eventually alter them and/or extract scene information with classic computer vision techniques. The aim is to study the relationship between real world captured and computer generated images and propose an acceptable solution for their blending.

\section{From Augmented reality to Augmented Virtuality}

As we speak of Virtual Reality environments we refer to an experience in which the user is fully immersed, therefore as much isolated from the real world. The perception of one’s self actually being in a different place gets deeper when involving more advanced sensory to track our movements and proxies like avatars to keep alive what is called in VR the "sense of presence", a critical term that expresses the longing deep connection with user perception and psychology \cite{vr_presence}. However, there are cases where we might want to take full advantage of the high level of interaction with a virtual environment without giving away the connection with the real world. Virtual reality is mainly meant to bring one’s perception in a controlled environment with its own rules, but what if we want to retain all the aspects of the environment surrounding us plus enhancing it with virtual elements? What if we want indeed to be brought in a different place, where everything is real except our own presence?

\begin{figure}
\centering
\includegraphics[width=7cm]{pictures/1st_and_ten}
%\caption{This caption is very long \newline ---in fact, it is so long that it doesn't fit on one line}
%\caption
%[1st line \newline 2nd line]
%{\begin{minipage}[t]{.8\linewidth}This is the caption \\This is the second line \end{minipage}}
\caption{The "1st and Ten" system, one of the first successful applications of augmented reality in sports (source: \href{http://www.howstuffworks.com/first-down-line.htm}{howstuffworks.com})}
\label{fig:rugby_ar}
\end{figure}

Augmented Reality explores ways of including virtual elements in the real world. Typically application’s aim in AR (as we will refer to from now on) span from showing additional information to actually place entire objects in the observed scene. The goal is not only keeping intact contact with reality but making it interactive, enhancing it with arbitrary content in the less obtrusive and more intuitive way possible. AR may or may not have reasons to track user pose or actions, but never involves the usage of avatars of any sort, since user is meant to feel exactly where he actually is.

Nowadays, any device that embeds a camera, a screen and a decent processing unit can feature AR applications, whether its computational complexity depends on features extraction \cite{link_google_translate_AR} or no extraction at all \cite{link_IKEA_AR}. However, even though availability and cost of the hardware makes them accessible to anyone, their use is not as much diffuse as one would expect. The reason is to find again in obtrusion and intuition, which brings us to transparency and integration. The real seamless integration between real and virtual is limited by the not-as-integration between individuals and devices; current systems are still far from perfect, and system designers typically end up making a number of application dependent trade off, going from simply info-graphic content such as sport games on TV or HUDs to more immersivity and interaction oriented applications such as medical surgery, complex machinery repair and other tasks that require a specific training.

\begin{figure}
\centering
\begin{subfigure}{0.49\textwidth}    
\centering
\includegraphics[scale=0.5]{pictures/titanium-a}
\end{subfigure}
\begin{subfigure}{0.49\textwidth}
\centering
\includegraphics[scale=0.5]{pictures/titanium-b}
\end{subfigure}
\caption{Titanium Strong Virtual Drift, virtual (left) and real (right) view. A very effective example of augmented virtuality (source: \href{https://www.youtube.com/watch?v=WJyG76Izk8M}{youtube.com})}
\end{figure}

On the other hand, some applications are meant exactly for hijacking individual perception, even if not entirely, and carefully control it. That is the case where user is still supposed to interact with reality in some way, but how and when is not up to him. This is the general concept for Mixed Reality (MR), placed midway between pure virtual and real; according to Milgram’s continuum \cite{milgram_continuum}, AR is indeed a form of MR in its most reality-oriented form. Moving towards MR we find quite well-known devices such as night or heat vision goggles: not so far away to require avatar representation but enough obtrusive to completely replace environment as we see it.

As we start to inject into the scene arbitrary information, which is not at all derived from real environment, we cross the boundary of Augmented Virtuality (AV). Roles are inverted: virtual experience can be somehow enhanced with outside information or content. Usually applications of this kind aim to keep contact with real world but also experiment with new ways of representing it. Also the user feels to be somewhere different from everyday world, but still are cases where avatars are not needed depending on the implementation: think for example to tele-operating a humanoid robot, where an avatar is not really necessary if the robot stays in the sight of the user, even better if his "virtual eyes" are on robots head (as in telerobotic surgery \cite{telepresence_intro}. This is not true in tele-presence applications, where real world interaction is mostly perceived as an enhancement of VR experience.

\begin{figure}
\centering
\includegraphics[width=10cm]{schemas/milgram_continuum_enhanced}
%\caption{This caption is very long \newline ---in fact, it is so long that it doesn't fit on one line}
%\caption
%[1st line \newline 2nd line]
%{\begin{minipage}[t]{.8\linewidth}This is the caption \\This is the second line \end{minipage}}
\caption{A graphical representation of Milgram's continuum (from: \href{http://proceedings.spiedigitallibrary.org/proceeding.aspx?articleid=981543}{Augmented Reality: A class of displays on the reality-virtuality continuum})}
\label{fig:milgram_continuum}
\end{figure}

\begin{figure}[h!]
\centering
\begin{subfigure}{0.49\textwidth}
\centering
\includegraphics[scale=0.13]{schemas/reality1}
\end{subfigure}
\begin{subfigure}{0.49\textwidth}
\centering
\includegraphics[scale=0.13]{schemas/virtualreality}
\end{subfigure} \\[1em]
\begin{subfigure}{0.49\textwidth}
\centering
\includegraphics[scale=0.13]{schemas/augmentedreality}
\end{subfigure}
\begin{subfigure}{0.49\textwidth}
\centering
\includegraphics[scale=0.13]{schemas/augmentedvirtuality}
\end{subfigure}
\caption{A simplified overview of mixed reality approaches. Standard arrows picture direct connections, while dashed arrows partial intervention. Their direction indicate the flow of information.}
\label{fig:mixed_reality}
\end{figure}

\newpage

\section{Optical and Video See-through devices} %limits and advantages
Devices offering a very good compromise between interactivity and transparency for AR/AV immersive applications are Head Mounted Displays (HMD), where images are projected directly in the user’s view. HMDs often include on-board sensors like gyroscopes and accelerometers or simply passive markers like US or IR to keep track of head position, orientation and speed so that images can be displayed accordingly \cite{tracking_AR}. As for the displaying part, two systems design have been proposed: video and optical see-through. Both have as image sources real and computer-generated world, but end up to be very different for technical reasons. In a work of Rolland-Fuchs \cite{optical_vs_video_st} a comprehensive taxonomy of video and optical see-through technology is given, through which we will now describe background and motivate the choice of our work.

A video HMD with no images from the real world is common in VR applications: its ability to block out the real world view thus cover the highest user’s field of view with the crispiest computer-generated image possible. Adding on-board image capture capability it can be considered a video see-through device, capable of both AR/AV applications. Experimenting with AV has virtually no limits, since other kind of sensors or even more cameras may be involved to show something different to the user, as soon as a good level of immersion is guaranteed. The same cannot be said for a canonical AR application: captured images should appear as close as real and feel natural to the user, suspending his instinctive disbelief that what he is seeing is not real. A series of optical limits come in play that simply cannot be surpassed by a simple screen, which is just a simplification of how images are perceived by the human eye. To address this lack, the most pursued strategy is providing additional hints to the brain by involving head motion and even eye gaze tracking to act on image behaviour. But even if a good compromise can be reached, it cannot be considered acceptable in the long run: human brain is as good at noticing imperfections as adapting to them. This constant effort generates an increasing amount of stress in the user that, depending on the subject, may experience from simple eye strain to temporary sickness \cite{virtual_sickness}.

Optical see-through devices systematically solve the problem, providing semi-transparent lens or screen that lets light in while computer-generated images are merged through a set of mirrors. The connection with reality is kept nearly intact, therefore no perception side-effects are present as soon as device is perfectly calibrated. The limit of optical HMDs is only technological. While they are the best option for AR applications, they still lack the capability of showing environment as only a completely transparent glass would allow or virtual content as only a completely opaque screen would do. Moreover, there is always a discrepancy between overlay (region where virtual and real information are superimposed) and peripheral FOV of the user, an issue video HMD not really have since everything else from the screen is blocked out. Cost and build limits still not allow to cover high field of views such as ones achieved by recent video counterparts like Oculus Rift \cite{oculus_rift}. In other words, optical see-through solutions are promising for a variety of applications whose working space is within arm reach and require reliability and minimum distraction. Even with the best performance possible, a video see-through headset can be dangerous in situations where seeing what is going on is vital.

Optical HMD is so far the candidate to be more present in everyday life in the future. As we already mentioned, its current obtrusive nature makes it hard, but recently a lot of effort has been spent in this direction. "Light" versions of optical HMD can be identified in early commercial products such as Google Glass \cite{google_glass} or the still unreleased Microsoft Hololens \cite{microsoft_hololens}, whose aim is supporting multi-purpose applications to supposedly revolutionize man-machine interfaces.

As for video HMD, it is still very used in VR and tele-presence/tele-operation applications \cite{telepresence_intro}. The current work has its starting point on its limits and possibilities. We found its features to be best for experimenting with different media and 3D content where integration with reality is a possibility not a necessity. Our aim is to build a basic AR video see-through setup that can allow users to experience both VR and AR seamlessly, with a constant focus on code reusability for future technology development (that will eventually overcome limits of both optical and video HMDs) and researchers to deeper understand their connection in human visual perception.


\begin{figure}
\centering 
\begin{subfigure}{0.49\textwidth}
\includegraphics[width=\linewidth]{schemas/videoST_hmd}
\end{subfigure}
\hspace{\fill}
\begin{subfigure}{0.49\textwidth}
\includegraphics[width=\linewidth]{schemas/opticalST_hmd}
\end{subfigure}
\vspace*{0.3cm} % (or whatever vertical separation you prefer)
\begin{subfigure}{0.49\textwidth}
\includegraphics[width=\linewidth]{pictures/videoST_example}
\end{subfigure}
\hspace{\fill}
\begin{subfigure}{0.49\textwidth}
\includegraphics[width=\linewidth]{pictures/opticalST_example}
\end{subfigure}
\vspace*{-3mm}
\caption{A video (top-left) and optical (top-right) see-through HMD conceptual diagram comparison for AR (source: \href{http://www.cs.unc.edu/~azuma/ARpresence.pdf}{A Survey of Augmented Reality}). At bottom two respective examples, WorldViz VideoVision powered HMD (left) and Simeye XL100A (right).}
\label{fig:video_optical_seethrough_comparison}
\end{figure}

\section{Perceptual goals and technical issues}
\subsection{High field of view}
As said, immersion in general depends on factors that allow user experience to be as close to real world perception as possible. FOV essentially limits head movements of the user, who can simply gaze at objects he manipulates in the natural way he is used to in everyday life. Restricting a person’s FOV, in fact, has been shown to affect people’s behaviour and degrade task performance \cite{restricting_FOV}. With the advent of modern VR HMDs, we cannot exempt from classifying high FOV as a concern. Though is not really necessary for a variety of applications, neglecting the full potential of recent hardware is to be considered a waste. Oculus Rift had its strong points on high FOV and low latency and is on the verge of releasing a product for the masses. For a VR/AR application, it will be required not only working at high frame rates with high image angles, but also filling discrepancies between virtual and real captured images. Even though user may adapt to one image with scale different from the one he is used to, he would not be able to (and must not be required to) decide between two. Apart for custom built solutions, FOV in HMD and camera never happen to match perfectly, so this must be kept in mind. This is a consequence of specific hardware implementations.

\subsection{Minimum distortion and high resolution}
Digital (as resolution) and optical (as distortion) per-device limits come in play, as technical and not perceptual issues like FOV discrepancy. All HMD setups involve a specific setup of mirrors/lenses and a consequently adequate screen/projector to get the image just right, no matter if optical or video. Resolution is a problem that is on the way of being solved with the advancements in displays, projectors and GPUs industry. Application is not really dependent on that except from the amount of computational load to be generated. By increasing FOV, resolution becomes more critical, as image dots must be spread across a wider area.
For high field of view HMDs, lens distortion and chromatic aberration are not negligible. "Undistortion" (a slang term indicating the distortion correction process) has to be performed in the rendering pipeline prior to visualization for both virtual and real image.

\subsection{Real and virtual perception discrepancy}
We already mentioned the need of matching FOV as a perceptual issue, way more noticeable when working with high angles. We classify under perceptual discrepancies all issues that would most likely destroy immersion. Dealing with an AR application with a video HMD, we focus on reaching the best possible imitation of what would offer an optical setup. Stereoscopy affirms that a stereo rig can provide binocular parallax, so that depth perception can be reached for both virtual and real objects. Moreover, raw depth information can be extracted from the real scene and simulate occlusion between real and virtual objects \cite{GPU_accel_stereo_AR}.

While this is feasible in practice, geometric problems keep us apart from delivering the desired effect to the user: a virtual object (detected or entirely generated) is placed in a 3D environment and its projection can be rendered relative to user's view position/orientation; a real scene image is instead already a projection from camera point of view and should in theory match with the virtual one. A good trick is to use mirrors, so that camera optical path reflects overlaps with eye's, even if in a different position \cite{optical_vs_video_st}. This solves the issue but at a great cost: camera FOV is sacrificed due to the path between the mirrors and is also the reason why optical see-through have this limit in the first place. When high FOV is a requirement and no mirror is involved, user is assumed to adapt to the shifted viewpoints of the cameras, so cameras are supposed to be placed as close to the eyes as possible.

The main perceptual limit for video and optical see-through is depth of field reproduction: while video HMD can only offer a geometrical simplification of real stereo image (since real cameras cannot be placed where eye pupils are) which is not a problem for the optical counterpart, simulating virtual stereo troubles both. This is because screens and projectors reproduce images on a flat surface at a constant distance from the observer, on which each eye has to focus on to get the image crisp. This is not really a problem when one non-stereo image is presented in front of both eyes since it is recognized flat as it is (like whatching a tv). As soon as stereo comes into play, objects can be moved and be perceived behind or in front of that surface and the brain will eventually ask extraocular muscles to converge eyes but also iris to focus to that (virtual) converging distance; light reaching each eye still comes from the actual screen, so there will be a discrepancy between eye convergence and expected focus distance, introducing strife and in the worst case object will still appear blurry. This is why, with current technology, it is good practice to place the display surface at the expected average working distance \cite{stereo_rules}, or at infinite if that is not known in advance where presumably there is minimal eye strain. This aspect is essentially independent from software VR/AR implementation and can be only partially countered (we will discuss such work-arounds in the next chapter) so we can only feel confident that this hardware limitation will be overcome in the near future (lightmapping technology \cite{light_field_mapping} and the to-be-unveiled "Magic Leap" project \cite{magic_leap} in Figure \ref{fig:magic_leap}).

\begin{figure}
\centering
\includegraphics[width=13cm]{pictures/magic_leap}
\caption{What should look like through Magic Leap technology \cite{magic_leap}, without the use of special effects or compositing. Note how virtual objects get in/out focus coherently with real environment (source: \href{https://www.youtube.com/watch?v=kw0-JRa9n94}{youtube.com})}
\label{fig:magic_leap}
\end{figure}

Other discrepancies regard detail of virtual objects, such as shadow and lights and realistic texturing. Users can still distinguish between a real object and its virtual counterpart depending on the fidelity with which it has been reproduced. This however depends on developer's choice: one may want to represent a real world augmented with evident digital cues (in order to keep user conscious of what is real and what is related to what is real) or focus on trick users in thinking all he sees is real (or vice-versa all is virtual). Our efforts in reducing discrepancies perception will concentrate on keeping whatever real or virtual content consistent with whatever scene is shown on screen. This goes for perfectly overlap virtual and real worlds and, as an eventual future development stage, their deeper interaction with depth mapping.

\subsection{Performance limitations}
Fluent image rendering is the sine-qua-non for all previous conditions and can be achieved with reasonable (in gaming, 60 fps is considered the ideal standard) frame rate by any good 3D engine and satisfactory hardware for a VR-only environment. 3D graphics computation are exceptionally fast with dedicated GPU, but still require a good CPU for sequential operations. As we approach to computer vision applications, high parallelism is a desirable factor but nothing a modern CPU cannot handle for 2D computations (such as image undistort, object detection and so on). In VR/AR application, vision and rendering must then act in parallel and in real-time.
\iffalse
\\
TO EDIT\\
Other than disposing of top-notch hardware, there are a few rules of thumb to follow:
- lower image resolution: this would reduce computational load substantially. Always consider hardware and human limits unnecessary high quality settings.
- reduce frame rate: same considerations made above.
- simplify rendered scene (3D only): (to change) keep virtual objects at minimum number/complexity, crop them out if not visible again for limitations.
- 2D image elaboration through 3D hacks (to change)
- implement algorithms with OpenCL/CUDA: specific languages are available to parallelize custom operations on highly parallel hardware, but requires specific knowledge and dedicated implementation.
- run serialized code when CPU should be waiting for GPU\\
END EDIT\\
\fi
However, no matter what, inevitable to hit is the wall of "system latency", which refers to the time gap between a new set of data is retrieved from a sensor and the correspondent synthetic image is fully rendered. Since this exchange can be instantaneous only in theory, such physic limitation cannot be simply solved by more powerful hardware. In VR/AR applications, software "hacks" can be used to tighten the gap enough to trick the brain, as we will see. Moreover, since measure data belong to different natures, different non-synchronized devices working at different data rates, lag problems will be faced separately. 

\section{Document outline}
In the introduction we have provided a motivations for experimenting with AR on a video see-through HMD and a fast glimpse of the problems we will have to face in the following dissertation. In chapter 2, we will run deeper through aspects critical for understanding the options and state-of-the-art tools our work will exploit to tackle such problems. Once an overview for both reality and virtual reality has been given, we will focus on the proposed methodology in chapter 3, combining and pushing forward what presented in chapter 2. In chapter 4 and 5 we offer our technical results on tests performed and a conclusion opening to future developments and works.

\chapter{Related work and state of the art}
A VR/AR setup/application must deal with different aspects. Three are the main fields of expertise:
computer vision: describes the techniques for displaying/enhancing images and extrapolating information by means of imaging sensors as cameras according to different models. Stereo vision addresses a specific set of  problems regarding the use of two cameras in order to achieve more information and the stereoscopic effect similar to the human vision.
virtual reality: models mathematically the real world and aims to reproduce that feel in human users. It comprehends also techniques for image enhancement, but in this case images are entirely generated by the model. Also in this case a model is used for a camera observing a scene, even though fictional.


\chapter{Proposed solution}

\section{Preemptive overview}
In this chapter we face the challenges of reaching a good compromise between stereo augmented reality and immersive VR in one application. To every issue early mentioned in the introduction, one by one, it will be presented an answer in the following paragraphs. More in detail:
\begin{itemize}
\item gear selection and stereo rig design (high field of view and high resolution requirements);
\item image mapping and stereo augmented reality (minimum distortion and facing real and virtual discrepancy in stereo);
\item latency compensation (other discrepancies induced by technical and physical limitations);
\item implementation (all the above with considerations on performance).
\end{itemize}
Will follow in next chapter observation of work's results that could be achieved with the chosen starting gear and deducing possible future developments in the conclusions.

\section{Gear selection}
We preferred to mention first the choices in hardware and software involved in the development of the project; this way we will be able to address the problems one by one and in parallel use our configuration as a practical example. We chose products available to the public that fits best our requirements within reasonable budgets. Software libraries involved focus on simplicity, future integration and come in part as a consequence of some of chosen assets. The table in [figure N] matches hardware/software used with the project goals anticipated so far.

\begin{table}[]
\centering
\begin{tabular}{llll}
\hline
Gear type                       & Model                                & Goals coverage                                                                                                & Limitations                                                                                            \\ \hline
\multicolumn{1}{|l|}{HMD}       & \multicolumn{1}{l|}{Oculus Rift DK2} & \multicolumn{1}{p{0.4\linewidth}|}{
- Low latency \newline
- High refresh rate (up to 75Hz) \newline
- High FOV (up to 110 degrees on horizontal)\newline
- Head pose tracking
}                                                & \multicolumn{1}{p{0.2\linewidth}|}{No documentation on how VR stereo is performed on the eyes}                        \\ \hline
\multicolumn{1}{|l|}{Stereo Rig}    & \multicolumn{1}{l|}{2x Logitech C920}   & \multicolumn{1}{p{0.4\linewidth}|}{
- High framerate (30fps)\newline
- FullHD resolution (1920x1080)
}                                            & \multicolumn{1}{p{0.2\linewidth}|}{Usb2.0 limits bandwidth, frames must be decoded from h.264}                        \\ \hline
\multicolumn{1}{|p{0.1\linewidth}|}{Computer Vision library}    & \multicolumn{1}{l|}{OpenCV 2.4.9}    & \multicolumn{1}{p{0.4\linewidth}|}{
- Camera calibration \newline
- Image undistortion \newline
- Stereo vision \newline
- Performance optimization (CUDA and TBB support)
} & \multicolumn{1}{p{0.2\linewidth}|}{Still a CPU-bound vision library, only partial CUDA support}                                                  \\ \hline
\multicolumn{1}{|l|}{3D Engine} & \multicolumn{1}{l|}{Ogre3D 1.9}      & \multicolumn{1}{p{0.4\linewidth}|}{
- Portability \newline
- Access to rendering pipeline code (open source) \newline
- Shading support \newline
- Compatibility with ROS
}   & \multicolumn{1}{p{0.2\linewidth}|}{Lacks of dedicated development and debug environment (not a technical limitation)} \\ \hline
\end{tabular}
\caption{Hardware and software tools chosen for the experiments.}
\label{gear_table}
\end{table}

The step by step followed for design is:
\begin{itemize}
\item identify class of HMD to build (type of components and its limits);
\item choose lens/mirrors according to focus and FOV to achieve, with acceptable or minimal distortion;
\item choose screen/projector basing on luminosity/contrast and resolution to provide enough detailed images given those lens/mirrors and with a target refresh rate;
\item choose camera (if involved for AR) so that image is closest match for previously defined focus, FOV and resolution;
\item consider additional or replacement lenses to cover needed FOV;
\item account for correcting camera lens distortion (necessary for detection, otherwise not needed if not noticeable by the user);
\item account for blending real and virtual image;
\item account for HMD lens/mirrors distortion (if not noticeable by the user).
\end{itemize}

\iffalse
\subsection{HMD and head tracking}

\subsection{Camera for stereo}
low cost
best resolution/fps rate

\subsection{3D Engine for VR environment}
OculusSDK
Ogre3D

\subsection{Libraries for CV and AR}
OpenCV + tbb + cuda
arUco
\fi

\section{Stereo Rig design}
Assumption is that the two cameras have same technical characteristics and no distortion (or were previously calibrated). According to Paul Burke [24], a good rule of thumb for building up stereoscopic camera rig is to choose camera aperture and focal lenght first, depending on the the distance from the lens we decide to be on focus. Supposing we have a planar display, it is convenient to keep the zero-parallax plane at the same distance from the viewer, so that convergence and accomodation will keep coherent when watching the part of the scene on focus; this implies that to get convergence right with a parallel setup, one could shift the cameras, the lenses or later the image on the screen, with all the consequences we already discussed. In the case of a toed-in setup, convergence is immediately defined by their tilting angle, with no need for subsequent image trim but at the cost of keystoning.

One great advantage in using a HMD with higher FOV than the cameras is that it is not really a problem to translate the image: once each camera's FOV is correctly mapped into user's view plane, each frame can be shifted or scaled without subsequent loss of FOV. This allows to dynamically adjust convergence after capture, for any stereo rig configuration. Moreover, the nature of 3D environment gives us the ability to easily correct keystoning, by either rotating or scaling the plane image it is projected into, instead of applying such transformation on the image [25]. The main reason left to favour a toed-in over a parallel setup for video see-through in high FOV HMD is the limits of the area covered by the insersection of the frustrums of each camera. If we plan to get stereoscopic effect on objects very close to the user, those objects also must fall in both frustrums. Excluding the more complex off-axis setup, this is only feasible in a parallel setup by raising cameras' FOV, which means switching lenses and introducing higher distortion, or manually tilting them inwards.

\begin{figure} 
\centering   
\begin{minipage}[t]{0.49\textwidth}
\includegraphics[width=\linewidth, height=7.4cm]{schemas/stereocam_only}
\end{minipage}
%\hspace{\fill}
\begin{minipage}[t]{0.49\textwidth}
\includegraphics[width=\linewidth, height=7.4cm]{schemas/eyes_only}
\end{minipage}
%\vspace*{2mm}
\caption{Having a stereo camera rig in toe-in configuration (left) has the effect to reproduce the exact depth distance of a target T on a screen (right) where eye-to-eye (OO') and eye-to-screen distances match inter-camera (CC') and zero-parallax distance. Note that the schema ignores keystoning effect, so the geometry is for all stereo configurations where zero-parallax plane is not at infinite}
\label{fig:stereo_eye_camera_comparison}
\end{figure}

Another common norm for stereo video footage is to use 1/30 of the zero-parallax distance as a measure for camera separation [26]. In first-person gaming, one commonly used strategy is to use the height of the camera from the virtual ground as zero-parallax distance and then calculate suggested separation: the result is a rough approximation of the proportions we have in real life. In our case, however, we prefer to keep camera separation as close to IPD: we want cameras to act as user's eyes and images to be perceived as close as reality in manipulation tasks; we will see how this choice will be fundamental for real-virtual matching in later paragraphs. While for footage on a screen the zero-parallax distance can be chosen so that we have the desired depth effect, in our case we will have to match that perceived distance with the virtual world, which will have its own stereo setup.

As for stereo computer vision analysis, stereo calibration can be performed and algorithms can run independently from how images will be displayed to the user. It is anyhow a crucial point to examine in depth how displayed image and the information extracted from it are kept consistent in the 3D environment and in which way they can interact. Assuming we perform detection and place a virtual object in the VR scene accordingly, how can we guarantee that it will be overlapped to the image in the right spot at all times?

The ideal case would be having both virtual and real stereo rigs in off-axis configuration, having real cameras coincident with real pupils. Proposed setup uses off-axis configuration for virtual rig, as also suggested by Oculus documentation and provided by SDK, and toe-in and parallel for the physical rig; in the specific, for standard and wide-angle type lenses (modeled with pinhole model) cameras are toed-in, for fisheye type lenses instead they are parallel. We will express the reasons for this choice in the following paragraphs.

To keep each camera closest to each eye and minimize virtual sickness [27], constraints implied for the physical cameras are:
\begin{itemize}
\item keeping them at IPD distance (average human reference is 0.064m);
\item placing rig at minimum distance from the eyes allowed by the HMD.
\end{itemize}
The configuration chosen to conduct our experiments is with cameras placed directly in front of user's eyes.

\subsection{Wide-angle lens configuration}
Wide-angle lenses, even with high distortion, still obey to pin-hole camera model. Horizontal FOV for such lenses may vary from 45-50 to 90 degrees with lens distortion increasing ad the edges. This category can partially or completely cover horizontal FOV of a video HMD. One more problem is that these devices declare a vertical FOV that is valid only when the whole sensor surface is used. This is always the case when capturing one single image: it will be memorized in camera buffer and then transmitted to whatever machine requiring it, in an offline fashion. When requiring a constant stream of high definition images, however, device meets buffering, encoding and trasmission speed limitations. For istance, our model is a high end consumer usb 2.0 webcam offering up to 1920x1080 (16:9) resolution at a stable 30 fps stream; to achieve this result, our camera not only uses a fraction of the sensor (actual sensor is a 2304x1536 (4:3) so it uses the widest subset of pixels forming a 16:9 image for live capture), but also downscales and encodes in h264 captured frames. Similar performances in usb 3.0 devices leads to much higher cost for industrial applications [28,29].

\begin{figure}
\centering
\includegraphics[width=\linewidth]{schemas/diagonal_FOV}
\caption{An overview on how a 16:9 image is extrapolated from a 4:3 sensor and vice-versa (left). Retailers often do not specify which ratio the declared diagonal FOV corresponds to, even though there is always a discrepancy on the vertical and horizontal (right). (source: \href{http://therandomlab.blogspot.se/2013/03/logitech-c920-and-c910-fields-of-view.html}{http://therandomlab.blogspot.se})}
\label{fig:timewarp_timeline}
\end{figure}

Such frame displayed in front of one eye in the rift covers most of its horizontal range, but only about a half of vertical. A solution to this problem has been proposed by William Steptoe [30]: first match per-eye resolution of the HMD with the sensor's, then replace current lens with one with smaller focal length. This implies rotating each camera by 90 degrees, so that the new 9:16 configuration matches closely Oculus Rift DK2 per eye resolution of 960x1080. This is true, although lenses allowing to match camera vertical FOV with Oculus horizontal are very hard to find and would introduce too high distortion and information loss at the edges; it gets indeed coverage of DK1 vertical FOV but was it does not exploit all its horizontal capabilities (figure). Furthermore, there are two major reasons why we chose to stick with horizontal cameras:
\begin{itemize}
\item even assuming possible to cover eye horizontal angles with rotated sensor, there would be a substantial image loss on the vertical, since camera horizontal FOV would have reached much more than needed angle;
\item it is more likely to the user to rotate his head and eyes to the right and left; with non-rotated sensor, stock lens is enough to cover each eye's horizontal, while an enhanced lens can improve vertical coverage and the portion of horizontal out of the border is still useful information that can cope with capture-display latency (discussed in later paragraphs);
\item stock lens provides an efficient mechanism of auto-focus that comes in handy with up-close real object observation; note that focus of each camera is set by default at infinite: due to the short focal length typical of webcams (that usually require high field of view) everything further than N meters is on focus. For closer objects, autofocus can be activated, but both focal length values must keep in sync so that there is no discrepancy in blurriness between the two images.
\end{itemize}
In such limited FOV conditions we opted for horizontal cameras in toe-in configuration. Cameras converge at 40cm from the user, which is our estimated average for basic hand manipulation so tasks. Stereopsis can be achieved for very close objects up to N cm from the user in negative parallax. Instead of a tunnel-like view effect for FOVs insufficiently high in respect to human eye, we get more a window-like view: HMD provides a view range wide enough for user's eyes, but camera image boundaries are visible, implying that real objects are perceived as coming in or out a "window", whose distance corresponds to the zero-parallax. Arms in view closer than this distance are seen as "cropped out" from the bottom, which clearly shows negative parallax limits. Ideally, this window should be moved as close as possible to the eyes (shouldn't be the HMD itself the supposed "window"?) but this is not really practical nor suggested for all the issues discussed above. Note that this fictional plane is only an optical perception: the distance at which the actual virtual object onto which image is projected can be arbitrarily and dynamically changed. This won't matter where real objects will be perceived, but where virtual object will be clipped by such plane. One advantage is the chance to decide whether the image should appear in front or behind static objects in the scene (such as virtual floors and walls) by changing its distance (manually or automatically) in respect to the z-value of such objects instead of acting on the render order.

\iffalse
TOED-IN
(-) edge violation, "see through a window" effect (lower immersion)
(-) keystoning effect at edges, but reducible
(+) compensate the low stereo cues area of parallel config
(+) easy to implement
(+) good vertical disparities (VSR)
\fi

\subsection{Fish-eye lens configuration}
Way more high FOV can be gathered by fisheye lenses, which can cover angles from 120 to 170 degrees, up to 190 in extreme conditions. Inevitably, lens distortion is also very high, but responds to a different camera model, where there is no unique optical axis, but infinite spanning from lens focal center to every point of its spherical surface [31]. This is way image captured from a planar sensor from a fisheye lens model appears as a circle, with almost no distortion at its center, compressing toward the edge until disappearing on the boundary: this is what we get when projecting a spherical surface on a plane. This is addressed as epipolar or stereographic projection in geometry and is widely used in fields like geography and cartography [32] where the need to map targets non-planar by nature into planar representations is met.

For computer vision, but also for a naked eye, such images need to be undistorted to see the scene in the proper way. On a 2D plane, by undistortion we intend a sequence of transformations or reprojections that returns, in the end, a 2D picture. For what we said about the capabilities of a fisheye lens, this shows a clear limit: it requires a planar surface of infinite dimensions to reproduce 180 degrees of FOV correctly, and still unpractical proportions for values as high as the human eye's.

(figure)

Our solution borrows the techniques used in immersive environment based upon hemispherical domes [33]. In particular, how omnidirectional images can be displayed so that the combination of viewer position and display itself perform the undistortion we are looking for: in principles, reverting a stereographic projection from the plane, the result of fisheye lens capture on the planar sensor, to its original 3D representation. Since we have two capture devices, what we also need is a stereo configuration to achieve stereopsis with such image pairs. Supposing we have a viewer in the center of a physical semi-spherical dome, capable of stereo, with eyes looking straight at its middle; the simplest configuration would be rotating each image, meant for each eye,  virtually inwards such that zero parallax occurs at the correct distance along view direction and project them to the same surface [34]. This can correspond to our setup with toed-in cameras and fisheye lenses, where images are live captured and "virtual" rotation of the image corresponds to a real rotation of the sensors. But as Paul Bourke stresses, correct parallax information is returned by the sistem only when viewer's eyes are oriented straight forward, towards the region where considerations made for pinhole toed-in stereo apply. The alternative to this approach is called off-axis fisheye projection [35] which can be considered the counterpart of image displacement in planar displays: each omnidirectional image is virtually shifted and then reprojected on the dome, so that satisfactory depth sensation is formed for viewing directions different from the one mentioned above. Our camera setup can easily match this dome projection algorithm by shifting them perpendicularly to viewing direction, in a parallel configuration, which is our chosen option for capture FOV higher than what is allowed per-eye by the HMD.\\
For our experimental setup, we can take advantage of the virtual 3D environment to recreate the same conditions: a half sphere mesh, solidal with user's virtual head,  can be used as a dome by orienting face normals towards the center and using the raw fisheye image as a texture. To avoid complex shading algorithms, a pre-compiled UV map has been used and the chosen equipolar mapping method is shown in figure N.\\
Note how this solution fits elegantly to solve most of the problems encountered so far:
\begin{itemize}
\item in both omnidirectional offaxis and toe-in approach, viewer experiences a decrease in parallax information towards the edges of the dome (angles next to -90 and +90 in respect to its original view direction); this loss can only be appreciated in a real dome, where an individual can actually rotate his head in respect to it, meanwhile in our setup he is only able to rotate his eyes, which in turn cannot get disparity on its sight borders by nature; such extreme eye movement are also unlikely to happen in reality;
\item HMD allows us to return to each eye its own view, meaning a different dome and projected image; therefore both virtual eyes can be placed at the center of the semi-sphere at all times, an ideal and never verified condition for physical stereo dome environments;
\item the symmetric nature of the projection and its extended FOV can come in handy for compensating latency issued between image capture and head movement without the user noticing, as we will discuss in later paragraphs.
\end{itemize}

\iffalse
PARALLEL with high fov
(+) no edge violation (close objects are welcome, high immersion)
(+) no keystoning
(+) no need to toe-in to compensate low fov..
(-) .. but still can be very distorted at edge (need high quality lens!)
(-) no stereo cues at the edge of the image
(-) hard to implement (due to additional undistortion)
(+) good vertical disparities (VSR)
\fi

%\section{Application pipeline}


\section{Image mapping}
As ar interaction scene is finally rendered on a screen, a straightforward approach could be to first render the background or the skybox, then render correctly FOV mapped image and display it, then render virtual objects on top. This approach is what happens for AR applications displayed on a common display, where the actual background is the real image in all conditions. On a video HMD, however, the parameters according which image is finally shown to the user can be much more complex than screen dimensions and resolution: mirrors or other optical solutions are involved to cover the needed FOV, meaning that the image on the screen must cope with introduced distortions, much like what Oculus Rift does. The mapping between HMD and the screen and the system implementing it is device-dependent: such information can be gathered only from documentation or from SDK. Moreover, any further adjustment on the image must be carefully implemented by tweaking the projection formula used, which can be not very intuitive and error prone when editing it in code. When working with a 3D engine, where z-buffer and rendering is usually hidden, the real image overlay step must be implemented specifically for that developing environment.\\
Our approach proposes a solution to manage the 2D image FOV mapping and reprojection on video HMD independent from its physical implementation: if the HMD is capable of rendering properly a 3D scene, then is also capable of blending a 2D camera image in it, whatever the FOV discrepancy.

\subsection{Pinhole camera model mapping}
In a 3D virtual environment, scene is rendered through a virtual camera, modeled by a view and perspective matrix: the first responsible for transforming vetrices coordinates from world to camera reference, the latter used for the actual 3D to 2D projection. For what it does, the latter is cousin of the intrinsic parameters matrix used in computer vision, but still differs in the form.\\
Hartley and Zisserman's intrinsic matrix [?] is parametrized as
\begin{equation}
	K = \left ( 
    		\begin{array}{ c c c}
    		f_x & s   & x_0 \\
    		0  & f_y & y_0 \\
    		0  & 0   & 1 \\
    		\end{array}
    \right )
\label{hz_intrinsic_matrix_original}
\end{equation}
being $f_{x}$ and $f_{y}$ the horizontal and vertical focal lenghts and $x_{0}$ $y_{0}$ the principal focal point offsets. The former determines the range of angles FOV covers while the latter decides the skew of the optical axis that may represent, for instance, shift of the sensor or the lens (done on purpose for off-axis configurations). We will not consider the skew $s$, which would describe a relative tilt of sensor in respect to the lens or viceversa, as it is not of interest in our work.

Projection operation is performed in computer graphics by the "projection matrix", documented as \ref{gl_proj_matrix_original}. Two conceptually different operations are performed:
\begin{itemize}
\item actual perspective projection (from 3D to 2D points);
\item conversion into normalized device coordinates (NDC) which is an implementation dependency of the rendering pipeline.
\end{itemize}
\begin{equation}
P = \left( \begin{array}{cccc} \frac{2 near}{right - left} & 0 & A & 0 \\ 0 & \frac{2 near}{top - bottom} & B & 0 \\ 0 & 0 & C & D \\ 0 & 0 & -1 & 0 \end{array} \right)
\label{gl_proj_matrix_original}
\end{equation}
where
\begin{equation}
\begin{array}{c}
A = \frac{right + left}{right - left} \\[0.6em]
B = \frac{top + bottom}{top - bottom} \\[0.6em]
C = -\frac{far + near}{far - near}  \\[0.6em]
D = -\frac{2 \; far \; near}{far - near}
\label{gl_proj_matrix_original_details}
\end{array}
\end{equation}
Note that the virtual reproduced FOV is expressed in terms of $right$, $left$, $top$, $bottom$ edges of a virtual image plane, which will result in the rendered frame. Also both projection matrix use a reference system that only differs from z-axis direction (Figure N).

\begin{figure} 
\centering   
\begin{minipage}[t]{0.45\textwidth}
\includegraphics[width=\linewidth, height=4.4cm]{schemas/hz_camera}
\end{minipage}
%\hspace{\fill}
\begin{minipage}[t]{0.45\textwidth}
\includegraphics[width=\linewidth, height=4.4cm]{schemas/gl_camera}
\end{minipage}
%\vspace*{2mm}
\caption{Pinhole camera model (left) and GL camera model (right) "looking" at opposite directions.}
\label{fig:stereo_eye_camera_comparison}
\end{figure}

Following the steps of Kyle Simek's work[?], intrinsic matrix (\ref{hz_intrinsic_matrix_original}) reference system can be then inverted on Z axis (by inverting sign of 3rd column) and can include additional z-depth information to match closer $P$ from \ref{gl_proj_matrix_original} closer:
\begin{equation}
K_{persp} = \left( \begin{array}{cccc} f_x & s & -x_0 & 0 \\ 0 & f_y & -y_0 & 0 \\ 0 & 0 & X & Y \\ 0 & 0 & -1 & 0 \end{array} \right)
\label{hz_intrinsic_matrix_extended}
\end{equation}
where
\begin{equation}
\begin{array}{c}
X = near + far \\
Y = near * far
\end{array}
\label{hz_intrinsic_matrix_extended_details}
\end{equation}
By introducing in \ref{gl_proj_matrix_original} focal lenghts different than $near$ by altering scale of virtual image plane
\begin{equation}
\begin{array}{c}
left' = (\frac{near}{\alpha}) left \\
right' = (\frac{near}{\alpha}) right \\
top' = (\frac{near}{\beta}) top \\
bottom' = (\frac{near}{\beta}) bottom
\end{array}
\label{near_alter_scale}
\end{equation}
and non-zero principal point offset by further translation of frame boundaries
\begin{equation}
\begin{array}{c}
left'' = left' - x_0 \\
right'' = right' - x_0 \\
top'' = top' - y_0 \\
bottom'' = bottom' - y_0
\end{array}
\label{near_alter_offset}
\end{equation}
we can simulate a general intrinsic camera matrix with zero skew $s$.

At this point, relevant camera characteristic parameters can be extracted from both matrices. The main difference lays in the model: "Normally, the film plane of a (real) camera is behind the focal point (i.e. the aperture) of the camera. The image can, however, be regarded as lying on a virtual image plane in front of the focal point. This arrangement is essentially the same as the virtual camera setup used in computer graphics rendering: the focal point is the "eye" point, the size and position of the image rectangle controls the field of view, and the perpendicular to the virtual image plane provides the "gaze" direction" [36]. This means that any HMD (or any other device displaying virtual imagery from 3D world) will most likely be using a virtual camera much like a real one; we say "most likely" because there do exist other less diffused proposed models for 3D rendering applications, such as Vrui [37].\\
Rethinking real camera in terms of image plane, we can port its FOV and image into the virtual camera with a planar mesh in front of it, perpendicular to its view vector, where size and distance from the virtual camera focal point are strictly correlated.

Consider its horizontal dimension $H_{dim}$ and its distance $d$ from the center of projection, so that it covers $H_{fov}$ of FOV angle; being
\begin{equation}
\alpha = \frac {H_{fov}} {2}
\end{equation}
by trigonometric construction we have
\begin{equation}
\tan \alpha = \frac {H_{dim}} {2 d}
\label{Hdim_relation}
\end{equation}
Consider now the relationship [n?] between horizontal FOV $H_{fov}$ of a pin-hole camera and its relationship with horizontal sensor dimension $s_{w}$ and focal lenght $f_x$: 
\begin{equation}
H_{fov} = 2 \arctan \frac {s_{w}} {2 f_x}
\label{Hfov_relation}
\end{equation}
By combining the \ref{Hdim_relation} with \ref{Hfov_relation}, we have 
\begin{equation}
H_{dim} = \frac {s_{w} d} {f_x}
\label{fov_distance_relation}
\end{equation}
The same can be verified with vertical counterparts $V_{dim}$, $s_{h}$ and $f_y$.

In other words, fixing a distance $d$ in \ref{fov_distance_relation} will determine horizontal and vertical size of the rectangle for the known real camera intrinsic parameters, and their ratio keeps constant when varying the distance. In fact, it does not make mathematically sense to define unique values for size and distance: virtual image plane is only an abstraction and can represent any parallel planar slice of the virtual camera frustrum, from distance "epsilon" to "infinite", so the projection can be generated by any of these positions. The reader can then ascertain that, by using a mesh as instance of the real image plane, camera FOV is not dependent from virtual camera FOV. HMD system can estabilish its own virtual camera parameters and mapping, covering a range of view angles in the virtual world, while real camera will have his own intrinsic camera matrix, with its own set of horizontal and vertical angles: as soon as real and virtual focal points overlap in the 3D scene, no discrepancy can be observed. An interesting observation is that we do not need to work with real unit values: calibration camera result may be expressed in pixels or meters but in the 3D scene they will still be expressed in virtual units (that can have any physical meaning) since all that matters is the ratio between those values.\\
One exception may arise from consistent off-axis configurations: if for either real or virtual camera focal point is off the optical axis, to keep real image plane centered it should be shifted sideways by the distance between the two positions. This happens because our model simplification that only considers distance on the view vector of the camera, ignoring oblique optical axis. Model can be enhanced by extracting focal points coordinates from the matrices and applying the offset to the plane, but since it would be just a slight amount for Oculus Rift virtual cameras and this is the same operation that can be manually performed for convergence adjustment, such exception is not caught in our implementation.

\subsection{Fisheye camera model mapping}
Considerations previously made for fisheye image projection find direct application in the method proposed for pinhole camera. Instead of using a plane, a half UV or polyshpere mesh is used in omnidirectional dome fashion, having its center coincident with focal point of the virtual camera (representing the eye). Distance and size condition is always preserved when varying sphere's scale symmetrically on the three axis (without deforming its original shape) since sphere has its pivot point in the virtual eye and covering 180 degrees in both horizontal and vertical direction.\\
There is a problem however with this setup: while the rectangular plane matches perfectly image to be projected in size, the sphere covers maximum view at all conditions, meaning that fisheye circle will very unlikely cover the whole surface, unless perfectly matching the FOV. Moreover, the way such image lays on the rectangular camera sensor is totally dependent on how lens is mounted and camera specifications. For this reason, once frame is taken, data must be mapped correctly on the mesh, which in turns means:
\begin{itemize}
\item placing the center of fisheye circle in front of user's view;
\item scaling the fisheye circle so that it covers the correct amount of FOV;
\item hiding the part of the image without information (hiding black areas out of the cirle).
\end{itemize}
The extraction of the fisheye circle could be performed by computer vision algorithms, detecting the black areas and determining its center: the latter to be performed at least once, while image cropping should be performed for each captured frame. To determine lens FOV, methods specifically meant for fisheye image calibration exist and are provided by common use computer vision libraries such as OpenCV [40] and Matlab [41]. Once FOV is known, we can initially assume it to be symmetrical and map fisheye circle on the dome surface. We will now see how a correct dome display of the image is straightforward with only means of UV map coordinates transformation.\\

Consider the result of a orthogonal stereographic projection of a half-sphere into a plane: it forms a perfect circle, having same diameter as the sphere. Assume now a pre-compiled UV map that implements the correspondence between points of the plane (an image) and the sphere surface (the dome mesh) from such projection. For simplicity, consider its initial center in $C^{map}_{u,v}=(c_x,c_y)=(0,0)$ and its rightmost and topmost boundary in $(1,0)$ and $(0,1)$ respectively, forming a perfect quarter of circonference on a perfectly square image and ray along $u$ and $v$ axes is $r^{map}_{u}=r^{map}_{v}=1$; this choice will help making next calculations more intuitive and independent from previous implementative choices of UV mapping.\\
Having identified the following data:
\begin{itemize}
\item image frame dimensions $w_{px}$,$h_{px}$ (in pixels)
\item fish-eye circle center coordinates in the image frame $C^{fsh}_{x,y}=(c_x,c_y)$ (in pixels)
\item fish-eye circle ray $r^{fsh}_{px}$ (in pixels)
\item fish-eye $fov_{deg}$ (in degrees)
\end{itemize}
we may want first to convert from pixels to UV space. Since $w_{px} \ne h_{px}$, $r^{fsh}_{px}$ will have two different dimensions along $u$ and $v$ axes:
\begin{equation}
\begin{array}{c}
r^{fsh}_{u}=\frac{r^{fsh}_{px}}{w_{px}}\\[0.6em]
r^{fsh}_{v}=\frac{r^{fsh}_{px}}{h_{px}}
\end{array}
\end{equation}
The same goes for $C^{fsh}_{x,y}$, where
\begin{equation}
\begin{array}{c}
C^{fsh}_{u}=\frac{C^{fsh}_{x}}{w_{px}}\\[0.6em]
C^{fsh}_{v}=\frac{C^{fsh}_{y}}{h_{px}}
\end{array}
\end{equation}
Since we chose $r^{map}_{u}=r^{map}_{v}=1$, we can easily rescale UV map points so that UV circle can match fish-eye circle dimension and consequently apply offset so that $C^{map}_{u,v} \equiv C^{fsh}_{u,v}$. In homogeneous coordinates:
\begin{equation}
T_{scale} = \left( \begin{array}{ccc}
r^{fsh}_{u} & 0 & 0 \\[0.5em]
0 & r^{fsh}_{v} & 0 \\[0.5em]
0 & 0 & 1
\end{array} \right)
\qquad
T_{offset} = \left( \begin{array}{ccc}
0 & 0 & C^{fsh}_{u} \\[0.5em]
0 & 0 & C^{fsh}_{v} \\[0.5em]
0 & 0 & 1
\end{array} \right)
\label{UV_scale_offset_180}
\end{equation}
With subsequent application of $T_{scale}$ and $T_{offset}$, the image circle is mapped on the half-sphere so that its surface is perfectly covered. This would be enough for correct mapping only if $fov_{deg} = 180^{\circ}$, which is the maximum coverage allowed by this projection and the dome. Since angle of view and angle on the sphere are proportional, UV scaling factors in \ref{UV_scale_offset_180} can be further reduced by the actual $fov_{deg}$ factor
\begin{equation}
\begin{array}{c}
r'^{fsh}_{u}=\frac{r^{fsh}_{u} \cdot fov_{deg}}{180^{\circ}}\\[0.6em]
r'^{fsh}_{v}=\frac{r^{fsh}_{v} \cdot fov_{deg}}{180^{\circ}}
\end{array}
\end{equation}
Since all the scaling is performed prior to translation, the additional factors can be merged in $T_{scale}$ by replacing the factors in \ref{UV_scale_offset_180} and resulting in a $T'_{scale}$.
As a result
\begin{equation}
\overbrace{
\left(\begin{array}{c}
u'\\
v'\\
1
\end{array}\right)
}^{\widetilde p'_{u,v}}=
\overbrace{
\left( \begin{array}{ccc}
r'^{fsh}_{u} & 0 & C^{fsh}_{u}\\[0.5em]
0 & r'^{fsh}_{v} & C^{fsh}_{v}\\[0.5em]
0 & 0 & 1
\end{array} \right)
}^{T=T'_{scale} \cdot T_{offset}}
 \cdot 
\xoverbrace{
\left(
\begin{array}{c}
u\\
v\\
1
\end{array}
\right)
}^{\widetilde p_{u,v}}
\label{UV_fisheye_map}
\end{equation}
which summerizes the dome fish-eye mapping with the suggested UV map start configuration.

With a generic UV map, getting the same result is trivial. It is sufficient to postpone to \ref{UV_fisheye_map} the necessary translation and scaling that transforms $C^{map}_{u,v} \longrightarrow (0,0)$ and $r^{map}_{u,v} \longrightarrow (1,1)$ so that they are performed in advance. The reader may solve the resulting combined transformation matrix for the generic case, that we prefer to keep separate.

By implementing the transformation as a fragment shader algorithm in sphere material, we have been able to reproduce its results; also by assigning zero alpha value to the area outside the circle (by either targeting black color values or better a binary mask) it is no longer necessary to preemptively crop each frame from the black area since such operation is a direct consequence of UV mapping.\\
In our application, fisheye image data extraction has not been implemented and is left for future development. However, by means of the shading algorithm, UV map offset and scale can be tweaked manually at runtime with a direct feedback to the user: such feature allows to display not only real video stream from the cameras, on which we have control for calibration, but also offline stereo omnidirectional imagery captured by unknown fisheye lenses.

\section{Implementing stereo augmented reality}
A premise of our pipeline design is to be able to perform computer vision or stereo computer vision independently from the method used to finally render images to the user. One may choose to enhance the frames and display such to the user or merely extracting information from one or both cameras leaving their view untouched. While the application unlocks such possibilities, our implementation focuses on coherent display of the two worlds, meaning blending (raw or filtered) real and virtual images and virtual content with real content, alias augmented reality. Technically speaking, real world content in a 3D virtual environment would classify such application as augmented virtuality, as we mentioned in the introduction; however, for this project, our final goal was not focus on VR (whose aspects eventually come in play) but to implement see-through and model an augmented reality headset through it.\\
Reader can agree that with sufficient knowledge in computer vision and shading scripting it is straightforward to alter camera image and render it to output; only limitations are performance and curiosity. The same cannot be said when working with virtual objects partially or entirely generated by information extracted from real images. In most elementary consumer AR applications [4,5], virtual entities are rendered on top of a real image which covers the entire screen, so that the gap between virtual and real is evident only for the limited size of the screen. In our case, image can cover in part or entirely the user's view: anything virtual placed on top of the image must persist in the 3D world he is experiencing, in or out the image. Such objects can be arbitrarly placed in the environment just like a VR-only application or placed somewhere according to what is sensed outside, exactly like position or orientation tracking sensors do to place user's avatar according to its real movements. With the use of cameras, we can take advantage of modern computer vision to experience what is seen by the user when the whole view around him is augmented.\\
Consider for simplicity an ideal camera, with no distortion and intrinsic matrix known, responsible to detect a the marker on the image. Knowing its physical dimension, it is possible to obtain position and orientation of the marker in respect to camera reference system. Such information can be directly used to place a virtual object in the 3D space, once real and virtual reference systems coincide. Given the considerations and the choices in setting up image projection mesh in respect to the virtual camera, there is no further transformation to be done: position and orientation of the real marker are only relative to real camera focal point and its orientation, whose model in turn overlaps with the virtual camera. This implies that by construction the projection of the marker will coincide with virtual object projection, as soon as coordinates of the latter are expressed in the virtual camera coordinate system. This has one relevant advantage: in our stereo rig, whether the marker is detected by the left camera and the entity position/orientation is set accordingly, this will appear also perfectly overlapped in right camera view since the 3D scene is shared between both eyes, as shown in picture (figure N). Stereoscopic perception is achieved by both objects seamlessly.\\
In practice, each frame from a real camera must be undistorted before working with detection algorithms and computer vision in general. After distortion correction, horizontal and vertical FOV and the projection of the focal point on the image plane is computed and put into the intrinsic matrix. Such information is then used to place the virtual projection mesh correctly in the scene, so that the focal point of real and virtual cameras will coincide, as previously said. The undistorted image, however, will be subjected to changes and marker detection (or any other elaboration) will be performed on that result. For a pinhole camera, the resulting image should be then projected in the 3D scene instead of the raw one, unless the distortion is far from noticeable: virtual objects will be correct anyway, but its background has to keep coherent in proportions. In our experiment, cameras presented low distortion at the edges so user could not tell the difference between using raw or rectified image.\\
Using a fisheye lens is instead a different argument: its pinhole camera equivalent is used to get a rectified result and intrinsic parameters, which is data used by computer vision, while the raw version is used in the 3D environment with its own undistort mechanism. Both return an undistorted view of the camera but it may not be straightforward how the overlap can be guaranteed. Again, all that matters is that focal points correspond and the undistortion is performed correctly in both cases. The computer vision algorithm may not work on the whole original FOV (figure N), but the detected extrinsics will still be expressed in respect to the focal point set in camera intrinsics meanwhile virtual camera can have its set of horizontal and vertical FOV values. Note that while for pinhole model projection mesh can be simply shifted to compensate errors in focal point correspondence, UV map coordinates must be shifted as well to achieve the same result with fisheye model mesh.\\
Some further considerations must be done for stereo convergence calibration: shifting inwards or outwards the two images (or any other translation) will disrupt such alignment. The previous explained implementation works for toed-in convergence and parallel fisheye, but will inevitably fail where further convergence adjustments are made in software. In this case, one may want to extend our geometrical model of the virtual stereo rig by putting a constraint on virtual object position and scale, so that it is also slightly shifted and returns to each eye the correct view, even though this can be seen as a "hack" because its final dimensions and coordinates will differ from what was originally percieved by the real cameras.\\
By means of head tracking, we can further improve the percieved detection performance: once an AR marker is first detected, its coordinates can be expressed from original virtual camera to virtual world coordinates and used to position the AR object permanently in the 3D scene; since camera image is in our setup dependent from HMD pose and user's head movements are mapped into virtual world, the correspondence between virtual object and marker will remain valid also when marker is not in view, as soon as marker has not moved. This is extremely useful in applications when there are both complete positional and orientation tracking and marker is not supposed to move [42], but still mitigates failures in detection, caused by motion blur (very likely since cameras are mounted on user's head)  or marker partially in view, and the only head orientation is used, as in our experiment. Virtual object will obey to virtual world physics until it is "synched" again with real world information. Other algorithms in need of such information could be optimized by using coordinates from virtual world which provides at best the real value and at worst a good approximation.

\section{Live capture latency compensation}
So far we presented an approach to reduce the perceptual gap between virtual world and virtualized real world through stereoscopy and VR, while reaching for distortion, resolution and FOV goals. We then implemented an AR example to demonstrate its convenience when dealing with real and virtual objects. Although these results are good from a vision point of view, latency is still a problem we did not tackle.
As mentioned in previous work, Oculus offers the timewarp mechanism as a mitigating solution for system latency between head movement and render, by reprojecting the view right before it is displayed to the eye. As a matter of fact, we have an additional problem of the same sort: the time of capture from each camera will always differ from the time it will be displayed in the scene, meaning we have a discrepancy in response between real and virtual.\\
The difference in frame rate is also a problem. An obvious choice is to choose camera with a capture rate as high as application render rate, but this can be again very expensive; the whole system frame rate could then be lowered down to camera's minimum, but this is a loss in immersion and rendering capability.
Timewarp can only work on a rendered image, covering the whole virtual eye FOV, and therefore cannot produce occlusion and out-of-border information; such limitations are however not noticeable since the entity of tranformation is minimized by the rendering pipeline and the head pose prediction algorithm. Such behaviour can be reproduced by exploiting the very same technique, but with no need of 2D reprojections thanks to our setup: inspired by timewarping solution, we can implement a mechanism for image stabilization and latency compensation to virtually solve both problems.

\subsection{Real-time image stabilization}
Consider our projection mesh and a camera with a lower frame than application is running at. It's position is relative to virtual camera and has the same orientation. Suppose the viewer is initially static and both 3D scene and real world image render coherent content (i.e. a real room with virtual objects); whenever user rotates his head, virtual head is updated as well (one time before each eye is rendered, to minimize latency) and the rendered views including the plane are sent to the timewarping algorithm. The render will show the new acquired view of the virtual scene, but camera image will still be there, in the same pose and position, and will keep this incoherent state until a new frame is captured and displayed. Hypothesize now we have a perfect camera capturing frames at infinite frequency. As soon as head rotates, we can get an update on both the 3D scene and the real world view, but both will not belong to the current instant of time due to system latency; however, while the former is entirely synthetic and such effect can be minimized by predicting pose values of the head, the latter offers only the rasterized view for a previous pose.\\
By augmenting the capture with pose information, we can reuse the obsolete image information and keep it consistent with virtual view: each frame can be placed in 3D space in the position it was originally captured, showing always the correct view of the world regardless of frame rate, system latency and user's head movements. Consider for instance the example in figure N, where a rendered view of a static virtual camera is projected into one eye, covering its FOV: at the beginning the eye is gazing straight ahead to the image plane showing a pinhole camera view; as soon as the plane orbits around virtual camera focal point, eye gaze follows it and, as soon as FOV mapping is done correctly, the projection of the plane into the eye remains the same.\\
Although this is true, assumptions for correct image projection may fall since real cameras do not rotate around the eye but orbit around the neck, meaning that the projection mesh should orbit around the virtual neck and not virtual camera focal point (figure N.1); such movement will disrupt the image projection condition where distance should remain constant during rotation. By adding such constraint (and therefore considering only orientation at which capture was taken, figure N.2) we can satisfy projection conditions, but this solution cannot consider in the equation the variation in parallax; while in fact in the 3D environment the two virtual eyes orbit around a virtual neck and observer can verify the expected parallax variation for virtual objects, the same cannot be reproduced simply transforming the projected image in space, since it contains parallax information from a previous viewpoint. This is exactly what happens for timewarping: reusing a previously rendered view and reprojecting it may return the impression user is actually rotating his view, but it has no way to reproduce the actual changes in occlusion given by the translation of each eye in a real head movement. This is still an unsolved problem in computer vision, since there is no way to "invent" information behind the objects in sight [?]; this is also why timewarping appears to work well for rotations (since time interval is short and the effect is less noticeable) but not translations of the head [?]. For now we propose this solution as an equivalent patch to timewarping that can improve experience under the same conditions: ideally, to have highest camera frame rate and lowest camera capture latency to confine occlusion errors into shortest intervals of time.\\
Expressing the camera and head orientations in respect to the same reference system, we consider the delta rotation expressed as follows:

(formula)

By applying such transformation to the projection mesh relatively to virtual eye (which is solidal with virtual head), we can fulfill pose independency between captured image and head rotation under the above conditions. Our demo implementation will also allow the to apply such transformation also in respect to head node (achieving the effect in figure N.1) to experience what discussed.

\subsection{Latency perception}
Even though the projected image is kept consistent with 3D scene and occlusion error is negligible, user will still perceive latency. This is because image is limited in FOV and a simple rotation cannot reproduce information that is still not there: in this case, we refer to angles of view that fall out of image FOV boundaries. For timewarp, image covers the whole view and "dark" areas show up at edges. For real camera image the effect is even worse, for the following reasons:
\begin{itemize}
\item camera image is refreshed at a way lower rate than 3D rendered view;
\item even at higher capture rates, system latency between head movement and a new capture displayed is much higher than between head movement and new 3D rendered view displayed;
\item camera image FOV can be lower than HMD FOV.
\end{itemize}
In other words, the viewer will always observe a delay in the frame "window" aligning to his change of viewing direction.\\
For timewarp, lack of border information can be compensated by increasing the FOV of virtual cameras: a wider render can cover more FOV than necessary so that no dark spots will be returned after reprojection. Such solution is not really used since it would impact on performance and the issue is not really noticeable in the first place.\\
On the other hand, when concerning images captured by physical devices the question is different. An increase in a lens FOV does not involve additional computational load and, distortions and resolution loss aside, can provide a valuable asset. Consider in particular the fisheye setup and its semi-spherical projection mesh: the higher the FOV and the lower distortion, wider the range within which frame rate and system latency conditions can potentially be compensated. Benefit is not limited to applications where vision system performance is scarce, but also where transmission of information intrinsically has delay, such as in teleoperation or teleassistance. An operator controlling a remote head or camera, for example, would neither need to wait the physical device to rotate or move them at all nor experience any delay when looking around, since once received image can be evaluated in an off-line fashion. This can be an object of future experimentation with our HMD setup.\\
Clearly, considerations made so fare are valid in terms of "space" but not of "time": the frame could be always placed in its correct pose in respect to the head and still cover the whole view, but frame rate and latency limitations will still show up as a delay in time in what regards the world surrounding the user. Hands, other people or any other thing in motion captured by the cameras will still appear as "happening too late", meaning it is still important to achieve maximum frame rate and minimum latency with the cameras. Conversely, a static scene will not return such perception, improving experience when exploring environments or making them interactive with the user's view.

\subsection{Time estimate limitations}
A critical assumtion was made so far: to know the exact pose at which the capture was made. Being able to do so is not trivial, unless the device itself is capable to fetch such information at capture time. A gyroscope is usually bundled in cameras with image stabilization, meaning that there is often additional hardware dedicated to read current pose and actively alter the body so that it compensates unwanted movements [?]. This is not what we look for, since we want to read the exact movements of the camera and then, possibly, use such information after capture. Moreover, HMD already provide pose estimation capabilities and since cameras are solidal with it, we already have access to that: given an interval of time, the practical issue faced is how to bond a captured image with the correct pose when pieces of information derive from different devices and system latency is an unknown variable.\\
A quantity that both measures have in common is time. Having at our disposal all possible poses occupied by head in time and the time at which a frame is captured, the correspondence can be found. This is however unfeasible for our application, since we encounter two major problems:
\begin{itemize}
\item the moment a new capture is asked to the camera driver or the moment it is returned never coincide with the time of the actual capture (for buffering and latency reasons); a timestamp retrieval or approximation mechanism must be devised;
\item saving all poses in time via software would require a dedicated and never preempted thread and would introduce consistent overload; the pose should be fetched in advance for the correct or expected time of capture.
\end{itemize}
A timestamping functionality can be implemented by the vendor, so that instant of frame capture can be known, once agreed between application's and camera's time reference relatively to which time is expressed. OpenCV does support timestamping retrieval from camera [post?], but at condition the used driver and the camera itself support it as well. We have implemented such feature, but this does not cover all cases.\\
Assuming absolute time reference is available, the easy alternative would be to fetch time value prior to capture request (in OpenCV performed by a blocking grab() call) or right after. We have observed however that, for video live capturing, the retrieval chain of camera/driver/OS introduces hidden buffering optimizations where the next frame capture is requested to the camera in advance (prior to next grab() call), apparently in order to minimize the waiting time upon request and the overall delay in retrieveing a new frame. Such feature is very welcome for a more fluent stream of data, but is a disturbance for our quest to locate the right instant in time. Our remark is of course limited to a specific setup, but introduces new considerations for the matter in general. From further inspection it appears that, isolating the setup from unpredictable disturbances on system, the delay between the actual capture and the request keeps in average constant in time. This claim comes of course from empirical and repeated measurements and is far from being able to be demonstrated; nonethless we were however to get a very close approximation, given a specific run of the application. Our implementation allows to set dynamically an offset delay value to subtract or add to the time value fetched at grab() call.\\
By the time timestamp is known, head pose for that time must be already been retrieved if actual capture happened in the past. We have seen how a future head pose can be extimated with time prediction algorithms to overcome the excessive latency experienced in HMDs, knowing current position and rotational speed. The approximation improves in quality when prediction time period is shorter. When asking for the current pose, the same latency considerations made for frame retrieval apply, but such undesired effect is compensated by optimized SDK implementation for the specific device and the less complex data to be transmitted, so that measure is returned almost immediately. Supposing one requests the current pose prior to grabbing a new frame, its pose estimation error would lay only in the capture latency error discussed above. On this matter, we propose two implementative solutions: once the delay in time between a new frame request and its actual capture has been extimated
\begin{itemize}
\item to request current pose a specified delay before every new frame request, by running a separate thread synchronously with the capture thread (figure N.1);
\item to extimate a past pose, by inverting head pose prediction algorithm and support past instants in time (figure N.2).
\end{itemize}
First solution will of course introduce overhead for context switching and also there is no guarantee that thread will be scheduled at the specified time, unless real time scheduling can be achieved on the target machine. The second, implemented in our version, uses the assumption that head rotational speed kept constant over time until "now" to "predict into the past" the pose that head might have assumed in the past; again, the approximation is very good when evaluating very short periods, like the case where actual capture time and grab() call are very close. The user may then experiment with the algorithm by tweaking the delay value until he percieves that real world scene reacts instantaneously to his head movements. For very high delay values, an hybrid of the first and second solution is suggested: even though thread may not perform request when needed, such smaller discrepancy can be still compensated by the past prediction algorithm.

\section{Proposed application pipeline}

(schema)

\subsection{3D scene rendering pipeline}

(schema)

Our application acts at two different 3D levels, or scenes as they are addressed in computer graphics:
\begin{itemize}
\item AR interaction scene: this level simulates all behaviours expected by real-virtual world interactions, whose implementation is device-independent. Briefly, this includes:
\begin{itemize}
\item user head/eyes virtual model (including capture-to-render latency compensation);
\item virtual objects, whose characteristics can be or not be extracted from real world;
\item real world image representation model (including its distortion);
\item skybox and other environment virtual support objects to improve experience;
\end{itemize}
\item HMD display scene: in this environment, image to be displayed on Oculus Rift screen is rendered from the previously mentioned scene, meaning that all HMD device-related distortion and optimizations are performed. In the specific:
\begin{itemize}
\item pincushion distortion, as opposed to barrel distortion caused by Oculus lenses, plus related chromatic aberration compensation for the ar interaction scene to be rendered correctly;
\item timewarping implementation, for render-to-eye latency compensation;
\item orthographic virtual cameras to match result with Rift actual screen.
\end{itemize}
\end{itemize}
Note that timewarping or other visual effects on 3D renderings are implemented with shading techniques. Effects on the 2D image, instead, will be applied separately from this pipeline.

\subsubsection{Enhanced head model}
The head model currently used for head-tracked stereo 3D applications can be enhanced with additional nodes (schema in figure N) which cover features discussed so far.\\
(schema)\\
Again, links in thick black represent parametric choices, manually adjustable or constant at runtime; thin lines define instead parameters managed by the application: each new delta stabilization is set to stabilization nodes relative to camera nodes, while AR object position and orientation is eventually set in world reference coordinate system by transforming its initial relative reference. Video nodes represent projection meshes, whose parameters are initially set by application on real camera parameters and can then be further adjusted by user. Each mesh is only rendered for the correspondent eye.

\subsubsection{Image projection model}
As we have seen in previous chapters, two virtual cameras are used to model user's view. In our implementation, IPD and ETN distance are kept in count, but its values must be manually set (in our case, are retrieved by user custom values provided by SDK). The enhanced model image reprojection allows to customize some of its relevant geometrical characteristics by means of dynamic adjustments. These include:
\begin{itemize}
\item shape: at current state, this can be switched between a plane (for pinhole toed-in camera configuration) and a half polysphere (for fisheye parallel configuration); a future implementation of vertex shader could generate a custom deformed mesh to correct more irregular distortions;
\item texture image: by default, the raw or computationally undistorted image of each camera is displayed, which is the critical case for this thesis; it is trivial to add support for showing offline content;
\item texture map: a custom implemented fragment shader allows the user to calibrate fisheye image reprojection on the dome, since this depends on the capturing lens/sensor used and an automatized implementation, if feasible, would arise application complexity; omnidirectional image circle can be translated and rescaled until presumed straight lines are percieved as straight in the headset;
\item position/rotation offset: convergence can be further adjusted by translating (in the pinhole case) or rotating (in the fisheye case) the shape, whose pivot point is the eye; as previously discussed, it is advised to act directly on the camera tilt when in toed-in configuration;
\item keystoning correction: for toed-in based configuration, both planes can be rotated in opposite directions further around their own vertical axis (perpendicular to its normal) so that its projection on virtual camera corrects unwanted keystoning effect; such option has not been implemented since effect was trascurable in our experiment;
\item virtual world clipping: scale and distance from the eye are strictly correlated to preserve the correct FOV mapping of the image; it is however possible to move forwards or backwards the projection mesh to clip virtual objects at the specified distance, which will fall in front or behind the image; we hope this aspect can be of inspiration for adding complexity with a future implementation of depth mapping with stereo computer vision;
\item "fictional" FOV adjustment: although not recommended, projected image can be deformed for both configurations to give the perception of an increment of FOV of the captured image itself; this is meant to compensate the fact that real cameras are not really placed on the eyes, but some centimeters far from them: as a result, observed real objects appear closer to how they should be.
\end{itemize}

\subsection{Image enhancement and vision pipeline}
Once a new frame is captured, image analysis and enhancement can be performed. Ideally, all computation should end before the next frame is requested; our application uses OpenCV which provides partial CUDA support.\\

(schema)

Undistort and arUco marker detection however can rely only on multi-thread CPU support (with TBB) so our implementation provides CUDA elaboration pipeline for image enhancement only, which runs in parallel with marker detection. To demonstrate its capabilities we implemented an existing GPU toon shading pipeline [Rifat Aras and Yuzhong Shen, GPU Accelerated Stylistic Augmented Reality] for AR plus the proposed Gooch shading for virtual object 3D rendering.

\subsection{Implementing time sync and pose estimation}
\iffalse
TIME\\
Approximate: simply get time before grab new frame\\
Precise-auto: compute image-to-grab delay from frame timestamp (if available)\\
Precise-manual: set image-to-grab delay manually\\
POSE\\
Approximate: get the current pose before grab new frame\\
Predicted: predict the pose in the past by a delay\\
Precise: get the current pose a delay before calling grab()\\
\fi
Application uses system clock, in respect to which both cameras and HMD are synchronized. HMD uses system clock to compute relative offsets in time for predicting head poses, which is the same used by application for image pose extimation and frame retrieval synchronization.\\
Each camera is managed by its own thread, who will wait for the next frame to be available and sends it to the computer vision pipeline. Frame synchronization is achieved by starting both threads with a common absolute time reference and frame frequency; a new stereo pair is then displayed when both threads signal a new capture is available. Both threads have responsibility to compute delay and jitter from last expected capture time and adjust sleeping time to catch up with next expected capture time.
Time reference is also used to extimate and express frame capture time. Provided modes to extimate capture pose are:
\begin{itemize}
\item None: no pose computation, camera image is always solidal with head;
\item Approximate: pose is fetched when a new frame is requested;
\item Predicted (auto/manual): pose is predicted a fixed delay before or after the new frame is requested (default);
\item Precise (auto/manual): pose is fetched a fixed delay before or after the new frame is requested (requires an additional deferred thread and has not been implemented).
\end{itemize}
At start, application tries to compute the necessary delay automatically, by requesting frame timestamp to camera driver; if such fails, a manual delay value can be dynamically adjusted by the user, which will be added or subtracted to time of frame request.


\chapter{Experimental results}

We will now present the results of narrow set of experiments aimed to observe both user confort when exposed to different visualization conditions offered by our application (chosen among all the possible setups provided) and to measure performance where the vision pipeline is active.

As a common setting, the 3D rendering engine has been forced to run at 60 fps while each camera runs at its best with 30 fps.  Performance results will be classified by computational hardware used, application setting and camera image resolution.

A consideration must be done for camera resolution. With cameras in stock configuration, the highest 16:9 resolution used in the experiments is 1024x576 (on the 1920x1080 available): the reason is that, since image will not cover the entire HMD FOV with stock wide-angle lenses, it is no use to use the whole sensor surface. Resolution was then further reduced whenever the subject could not perceive any loss in quality (thus improving performance at zero cost) or whenever we needed to experiment with performance to provide more complete results or a reasonable frame rate with hardware in use.\\
The high resolution of the sensor however still comes in handy when dealing with fish-eye lenses: to achieve the highest FOV possible, it is convenient to adjust the lens so that the entire fish-eye circle is restricted into the 16:9 area of the sensor used for video streaming, which in turn explains the need of higher resolution (image will need to be scaled to cover the entire HMD screen).

First tests were conducted on an Asus U500VZ, Intel i7 3632QM (four cores) CPU clocked at 2.8 Ghz and 8GB RAM, Nvidia GT650M with 2GB RAM.
Latest tests with CUDA and demo were conducted on "Aragorn" (from CVAP), Intel i7 3930K (six cores) CPU clocked at 3.20GHz and 32GB RAM, Nvidia Tesla K40c with 12GB RAM.

\section{Live stereo augmented reality demo}
In this experiment we tested the behaviour of the application in wide-angle toed-in configuration, streaming live images from the cameras into the HMD, and presenting them to the user. In this demo we feature a version of the application meant to study the impact of AR video see-through (in our specific setup) on multiple subjects, volunteering out of curiosity or sincere interest in the project, and to test the model and architecture capabilities. Features provided were:
\begin{itemize}
\item camera pose stabilization: can be turned on and off to test its impact on the subject;
\item skybox: meant to test discrepancy between head movements and image delay plus enhancing comfort, also can be turned on and off;
\item camera image undistortion and marker detection for augmented reality: demonstration of discrepancy reduction in depth perception between virtual and real objects and highlighted its limits;
\item "toon shading" effect: showed off CUDA filtering capabilities and impact on performance and user perception in Non-PhotoRealistic rendering conditions.
\end{itemize}

To 20 subjects has been asked to walk around in a room with HMD on and to provide feedback, test demo configurations and experience 3D perception when watching the marker static or manipulating it. To better understand how video see-through and immersive AR experience can be improved, the application has been structured in steps (figure \ref{demo_steps}). The first step has four configurations, starting from the bare minimum image visualization: subjects can freely switch between each other on personal preference and provide personal opinion and express discomfort, if any. Users were specifically asked to rotate their head around to identify the difference between stabilized and non-stabilized mode and then move on with activating the skybox.

\begin{figure}
\centering
\includegraphics[width=\linewidth]{schemas/demosteps}
\caption{Graphical representation of steps of the experiment. Arrows indicate which path and in which direction can be taken from a state to another.}
\label{fig:demo_steps}
\end{figure}

Once most confortable setup is chosen, the marker is placed in view and is given to the user the chance to manipulate it. As a result, previously mentioned geometrical problems revealed themselves in form of AR virtual object and AR marker mismatches. This offered a good excuse to play around with base application settings, meant exacly to overcome such limitations. With the help of users' feedback we could confirm that for every distance between actor and marker existed a set of settings that could solve virtual/real discrepancy for that distance. In particular, the setting that allowed to adjust perceived zero-parallax distance (by toeing in the image planes) was effective. With an on-the-fly hack was able to mitigate the problem by dynamically adjusting the plane toe-in angle in respect to detected distance of the marker from camera used for AR by linear approximation. Keystoning was not really noticed by anyone until problem was pointed out by us.

As final bonus, the effect pipeline was activated (with toon shading embedded in the demo). Final resolution of “cartoonized” camera image was chosen so that it was the closest to the original and achieved lower framerate. This way, subjects could both experience NPR environment and be exposed to the impact on performance with the chosen configuration. As algorithm stabilization still worked well, since it merely reproduces original image pose and time capture estimation did not change, but all subject noticed image plane additional jerking and found it discomforting. However, we do not suggest to implement any kind of interpolation: this would make plane move more smoothly, but the image will still remain static and the mismatch we worked hard to remove will again be there. Instead, cameras with a higher FOV should be used in order to hide completely image borders.

We have classified (in figure \ref{demo_subject_class}) the users which span from highly sensitive (which means they have preemptively declared they had past experiences with HMDs and VR or stereo content, not satisfactory) to people entirely new to the subject.

\begin{table}[]
\centering
\label{demo_subject_classes}
\begin{tabular}{llll}
Class 1                                                                          & Class 2                                                                          & Class 3                                                                           & Class 4                                                                          \\
\multicolumn{1}{c}{\cellcolor[HTML]{FFBCA4}{\color[HTML]{E10000} \textbf{3/20}}} & \multicolumn{1}{c}{\cellcolor[HTML]{FFD453}{\color[HTML]{F56B00} \textbf{4/20}}} & \multicolumn{1}{c}{\cellcolor[HTML]{9AFF99}{\color[HTML]{009901} \textbf{10/20}}} & \multicolumn{1}{c}{\cellcolor[HTML]{ECF4FF}{\color[HTML]{6434FC} \textbf{3/20}}}
\end{tabular}
\caption{Subjects classified on previous experience with Oculus Rift. Class 1: have tried and experience sickness; Class 2: have tried and could withstand; Class 3: have never tried, but report to be gamers; Class 4: have never tried and have no gaming habits.}
\end{table}

The premise is the experiment could be considered a success whenever the majority of subjects voted for the expected preferred configuration. Keep in mind that, for technical limitations, the setup presented was using toed-in pin-hole camera model and wide-angle lenses, so not the best according to what discussed so far. The results are however still relevant considering that toe-in is still a common configuration, for stereo computer vision (reasonably) and stereo imaging reproduction (erroneously). Since a focus for future stereo computer vision algorithms development is given, we welcome the level of challenge of optical issues involved.

\begin{table}[]
\centering
\caption{Perceptual results from Step 1 of the experiment. Order expresses class and number how many subjects for each class selected that configuration.}
\label{perceptual_results}
\begin{tabular}{c|c|c}
           & Stabilization OFF                                              & Stabilization ON                                               \\ \hline
Skybox OFF & \begin{tabular}[c]{@{}c@{}}0/3\\ 1/4\\ 1/10\\ 0/3\end{tabular} & \begin{tabular}[c]{@{}c@{}}0/3\\ 0/4\\ 1/10\\ 1/3\end{tabular} \\ \hline
Skybox ON  & \begin{tabular}[c]{@{}c@{}}0/3\\ 0/4\\ 0/10\\ 0/3\end{tabular} & \begin{tabular}[c]{@{}c@{}}3/3\\ 3/4\\ 8/10\\ 2/3\end{tabular}
\end{tabular}
\end{table}

A table of perceptual results for first step is given in figure \ref{perceptual_results} in sample and class. Results demontrate how only a majority of the sample completely agree that the proposed improvements helped in feeling the right response between head movement and image, while the minority quite naturally disagreed, expressing discomfort when exposed to strategies implemented so far addressing them as "distracting elements". Few exceptions in the end did not feel the difference at all, until asked to force the hand on purpose. AR pose estimation is among the understandable complaints since it is prone to detection errors that depend on multiple factors out of our reach, like illumination, camera image quality or the effective mismatch between a real rotation of the cameras and a virtual rotation of the head. As for the other two aspects, further investigation indicates that discrepancy, of any entity, disrupts immersion: image plane is indeed limited and users could appreciate this aspect only after a rescaled image was presented to the eyes. At that point, we explained how higher FOV cameras or even better fish-eye lenses could cover the whole view, thus hiding edges and making skybox apparently not necessary.

Whether or not the stabilization step was preferred, we took care in explaining how such aspect could be appreciated: switching between stabilized skyboxed and non-skyboxed, most subjects (15/22) have been able to realize how discrepancy between image (laggy) and background (reactive) is tackled when focusing on the image content and not the frame bounaries. The same effect can also be noted on stabilized non-skyboxed, by focusing on a real object then turning head suddenly and see the reaction, and with AR virtual object when moving head fast, being able to "follow" better the underlaying marker.

Major issue encountered in experiments is discrepancy between marker pose and perceived virtual object position in space. We feel confident that this is a geometrical problem that can be formulated and can exploit the image plane meshes approach to correct undesired effect under current condition, even though it will inevitably fail with more markers to be detected at a time (which is the one in view to be used as stereo matching reference?). This gave us further motivation to our design choice, where cameras are toed-in for optimal stereo computer vision and the optimal working distance can be tweaked manually at runtime (instead of needing to toe-in cameras at a different angle).

Performance was less criticized. 3/22 subjects pointed out there were lag problems (excluding people that referred to camera stabilizaion as the latency issue itself) but not at the point application was unusable. Different question is the the toon filter applied: there was a noticeable drop in camera FPS (not influencing Ogre's FPS) which forced us to reduce its size for the image effects pipeline, while marker detection still uses the same. The choice of 640x360 satisfied everyone and did not look as a loss at all given the kind of elaboration to be done with it.

\section{CUDA pipeline performance with NPR filter}
In the demo from previous paragraph has been activated the image effect pipeline, which uses the camera image and CUDA support to apply a Non-Photorealistic-Reality (NPR) filter. Gooch shading was also used to apply a "toon" effect to the rest of the virtual environment.

In this case, any experimentation focused more on performance measurement than user experience, since deeper research and control over virtual environment rendering output is needed to achieve reliable results in terms of psychological perception. We asked anyway a less formal opinion on subjects under test on whether they could still distinguish real objects (from camera image) from the virtual object (from AR pipeline): as far as the virtual object overlayed on the image and the user was almost static, it seemed really hard to tell the difference under our conditions. Tests were performed with both OpenCV undistortion on and off. Undistortion, toghether with image effects and enhancement, should be implemented in CUDA. With our setup, arUco was successful in detecting the marker properly even without undistortion. In any case, the undistortion implemented is an optimized version \cite{link_optimized_undistort_opencv} In table \ref{performance_image} we report the times of execution for each processing stage and the achieved frame rates for two different cases. Note that at all cases, 3D rendering frame rate kept on stable 60 fps.

\begin{table}[]
\centering
\begin{tabular}{c|c|c|c|c}
                                                                    & \multicolumn{1}{l|}{1920x1080} & \multicolumn{1}{l|}{1024x576} & \multicolumn{1}{l|}{768x432} & \multicolumn{1}{l}{512x288} \\ \hline
\begin{tabular}[c]{@{}c@{}}cv::undistort\\ (optimized)\end{tabular} & 18ms                           & 11ms                          & 7ms                          & 5ms                         \\ \hline
aruco::detect                                                       & 28ms                           & 15-16ms                       & 11ms                         & 8ms                         \\ \hline
CUDA pipeline                                                       & 42ms                           & 24ms                          & 20ms                         & 16ms                        \\ \hline
FPS no/fx                                                           & 26-28fps                       & 30fps                         & 30fps                        & 30fps                       \\ \hline
FPS with/fx                                                         & 9-10fps                        & 17fps                         & 24fps                        & 28fps                      
\end{tabular}
\caption{All measurements are made on "Aragorn" and have to be intended as average and per each camera thread. The frame rate instead measure keeps count of both cameras running. The processor on which the test run supported multi-threading and TBB was used by OpenCV.}
\label{performance_image}
\end{table}

To run our demo for perceptual tests, we fixed cameras running at 1024x576 (also because without high FOV lenses, using more is a waste) and CUDA pipeline to work with a reduced version of 512x288 to keep performance reasonable (with the "toon" effect on, any reduction of resolution of source image is less noticeable anyways).


\section{Offline stereo fish-eye demo}
Since in our experiments we did not have reasonable control over lens replacement and positioning on stock cameras, we proposed an alternative able to test the capabilities of our fish-eye mesh undistortion algorithm implemented in a GLSL fragment shader. The demo application (if specified) can run in "Fish-eye" mode, bypassing the standard "Pin-hole" setting. The application loads by default a fish-eye stereo pair, to experiment with shader calibration. Manual operations possible are aspect-ratio, translation scale of the texture for each eye (details are described in the documentation), which dynamically adjust the relevant parameters mentioned in Chapter 3.
No experiments were issued in this modality, since stereo calibration, when performed manually, could lead to very personal results. Moreover, without marker detection or AR in general, it is impossible to experiment virtual-real discrepancies in this state, even though the same adjustments are allowed.





\chapter{Conclusions and future work}

\section{Optimizations for future applications}

\iffalse
\subsection{Employing ultra-wide Fish-eye lenses}
Increased percieved FOV \\
Decouple from camera FPS\\

\subsection{Low FPS: decreasing image resolution}

\subsection{Undistort and marker detection in CUDA}

\subsection{"Mono" mode}

\subsection{Virtual nose}
\fi

\section{Further development}

\iffalse
\subsection{Depth mapping with CUDA, stereo computer vision or depth sensor}

\subsection{Enhancing interactions with virtual hands with Leap Motion}

\subsection{Integration with ROS}
\fi



Discutere in questo capitolo come � stata progettata la soluzione al problema trattato nella tesi, indicando anche se sono stati valutati vari possibili approcci o soluzioni pre-esistenti e giustificando le proprie scelte. Descrivere quindi la soluzione vera e propria.

Nel caso sia stato sviluppato del software non triviale, � buona norma dedicargli tre sezioni:
\begin{itemize}
\item architettura dell'applicazione (interazioni con gli utenti e con altri sistemi, moduli logici, flussi dati interni ed esterni);
\item manuale dello sviluppatore (descrizione dei moduli, degli algoritmi, delle interfacce e delle strutture dati);
\item manuale utente (come installare ed usare il programma, interfacce, comandi, dati in input ed in output).
\end{itemize}
Nel caso di software molto voluminoso, queste tre sezioni possono diventare tre capitoli separati.

\chapter{Risultati}

Inserire in questo capitolo i risultati conseguiti, cercando di analizzarli -- se possibile -- in modo quantitativo.


\chapter{Conclusioni}

Qui si inseriscono brevi conclusioni sul lavoro svolto, senza ripetere inutilmente il sommario.

Si possono evidenziare i punti di forza e quelli di debolezza, nonch� i possibili sviluppi futuri o attivit� da svolgere per migliorare i risultati.

% print bibliography
% La bibliografia, da inserirsi solo se ci sono state citazioni.
% In questo caso ricordarsi che bisogna sempre elaborare due volte il file .TEX
% perch� la prima volta viene generata la bibliografia mentre la seconda volta viene inclusa

% NOTA: citare il DOI non � obbligatorio ma MOLTO desiderabile

%\begin{thebibliography}{9} % se ci sono meno di 10 citazioni
\begin{thebibliography}{99} % se ci sono da 10 a 99 citazioni

\bibitem{linkzeropoint}
Youtube - ``Zero Point", VR 360 film, \\ \href{https://www.youtube.com/watch?v=DsXEUPS2uss}{https://www.youtube.com/watch?v=DsXEUPS2uss}

\bibitem{link_starwars_trailer}
Facebook - ``Star Wars: The Force Awakens", Immersive 360 Teaser Trailer, \\ \href{https://www.facebook.com/StarWars/videos/1030579940326940/}{https://www.facebook.com/StarWars/videos/1030579940326940/}

\bibitem{vr_presence}
Larry F.~Hodges, Barbara O.~Rothbaum, R.~Kooperα,
D.~Opdyke, T.~Meyer, Johannes J. de Graaff,
James S. Williford, Max M. North,
``Presence as the defining factor in a VR application",
GVU Technical Report, GIT-GVU-94-06,
Georgia Institute of Technology (USA), 1994

\bibitem{link_google_translate_AR}
Google Play App Store - ``Google Translate", formerly World Lens, \\ \href{https://play.google.com/store/apps/details?id=com.google.android.apps.translate}{https://play.google.com/store/apps/details?id=com.google.android.apps.translate}

\bibitem{link_IKEA_AR}
Youtube - ``IKEA AR Catalogue", commercial, \\ \href{https://www.youtube.com/watch?v=vDNzTasuYEw}{https://www.youtube.com/watch?v=vDNzTasuYEw}



\bibitem{milgram_continuum}
P. Milgram, H. Takemura, A. Utsumi, F. Kishino,
``Augmented Reality: A class of displays on the reality-virtuality continuum",
Proc. SPIE 2351, Telemanipulator and Telepresence Technologies, 282 (December 21, 1995), \doi{10.1117/12.197321}

\bibitem{optical_vs_video_st}
J. P. Rolland, H. Fucks,
``Optical Versus Video See-Through Head-Mounted Displays in Medical Visualization",
Massachusetts Institute of Technology, June 2000, Vol. 9, No. 3, Pages 287-309,
\doi{10.1162/105474600566808}

-- stereo virtual sickness

\bibitem{restricting_FOV}
P. L. Alfano, G. F. Michel
``Restricting the field of view: perceptual and performance effects",
DePaul University, 1990, Perceptual and Motor Skills: Volume 70, Issue , pp. 35-45,
\doi{10.2466/pms.1990.70.1.35}

\bibitem{GPU_accel_stereo_AR}
M. Sizintsev, S. Kuthirummal, S. Samarasekera, R. Kumar, H. S. Sawhney, A. Chaudhry,
``GPU accelerated realtime stereo for augmented reality", 2010, In Proceedings Intl. Symp. 3D Data Processing, Visualization and Transmission (3DPVT)

-- 10

-- 11

-- magic leap

\bibitem{disparity_depth}
N. Quian,
``Binocular Disparity and the Perception of Depth",
Neuron, Elsevier, Volume 18, Issue 3, p359–368, March 1997,
\doi{10.1016/S0896-6273(00)81238-6}

\bibitem{book_cv}
R. Hartley and A. Zisserman,
``Multiple View Geometry in Computer Vision", Second Edition, Cambridge University Press, March 2004,
ISBN: 0521540518

\bibitem{stereoscopic_3D_acquisition}
M. Hasmanda, K. Riha,
``The Modelling of Stereoscopic 3D Scene Acquisition",
Brno University of Technology, April 2012, cited from paragraph 1.1

\bibitem{correct_stereo_pairs}
P. Burke,
``Creating correct stereo pairs from any raytracer",
SPIE Three Dimensional Imaging and Remote Sensing Imaging, Vol 902, pp 85, 2001

\bibitem{camera_convergence}
R. S. Allison,
``The Camera Convergence Problem Revisited",
Department of Computer Science and Centre for Vision Research, York
University, 2004,
\doi{10.1.1.145.9490}

\bibitem{dynamic_virtual_eye_convergence}
A. Sherstyuk, A. Dey, C. Sandor,
``Dynamic eye convergence for head-mounted displays improves user performance in virtual environments",
Proceedings of the ACM SIGGRAPH Symposium on Interactive 3D Graphics and Games. ACM, 2012



\bibitem{offaxis_frustrums}
P. Burke,
``Offaxis frustums: What are they and what are they good for?"
Centre For Astrophysics and Supercomputing, Swinburne University, HET409, September 2004

\bibitem{link_toein_diffused}
O. Kreylos,
``Good Stereo vs Bad Stereo", a brief story of most diffused stereo technique, 2012 \\ \href{http://doc-ok.org/?tag=lens-shift}{http://doc-ok.org/?tag=lens-shift}

\bibitem{link_arrift}
W. Steptoe,
``AR-Rift", Augmented reality project for NPR perception, 2014 \\ \href{http://willsteptoe.com/post/66968953089/ar-rift-part-1}{http://willsteptoe.com/post/66968953089/ar-rift-part-1}

\bibitem{link_stereo_tricks}
D. E. Simanek,
``Digital stereo photography tricks and effects" \\ \href{https://www.lhup.edu/~dsimanek/3d/stereo/tricks.htm}{https://www.lhup.edu/~dsimanek/3d/stereo/tricks.htm}
May, 2010

\bibitem{link_stereo_calib_example}
J. Rambhia,
``Stereo Calibration", an example of stereo calibration using OpenCV \\ \href{http://www.jayrambhia.com/blog/stereo-calibration/}{http://www.jayrambhia.com/blog/stereo-calibration/}
March, 2013

\bibitem{stereo_rectify_parallelise}
H. Su, B. W. He,
``Stereo rectification of calibrated image pairs based on geometric transformation",
I.J. Modern Education and Computer Science (MECS), 2011, 4, 17-24


 -- Fisheye stereo calibration and epipolar rectification

-- tobi

\bibitem{link_oculus_limits}
O. Kreylos
``A Closer Look at the Oculus Rift", March 2014, \\ \href{http://doc-ok.org/?p=756}{http://doc-ok.org/?p=756}

-- predictive tracking

\bibitem{oculus_prediction} --

\bibitem{latency_sinequanon}
M. Abrash,
``Latency: the sine qua non of AR and VR", December 2012 \\ \href{http://blogs.valvesoftware.com/abrash/latency-the-sine-qua-non-of-ar-and-vr}{http://blogs.valvesoftware.com/abrash/latency-the-sine-qua-non-of-ar-and-vr}

\bibitem{karmack_mitigation}
J. Carmack, ``Latency mitigation strategies", Feb. 2013 \\ Backup link: \href{https://www.twentymilliseconds.com/post/latency-mitigation-strategies/}{https://www.twentymilliseconds.com/post/latency-mitigation-strategies/}






\bibitem{stereo_pair_cameras}
P. Burke,
``Calculating Stereo Pairs", 1999 \\ \href{http://paulbourke.net/stereographics/stereorender/}{http://paulbourke.net/stereographics/stereorender/}

\bibitem{keystone_correction}
R. Sukthankar, R. Stockton, M. Mullin,
``Automatic Keystone Correction for Camera-assisted Presentation Interfaces",
Proceedings of International Conference on Multimedia Interfaces,
October, 2000

\bibitem{stereo_pairs_game}
P. Burke,
``Create side-by-side stereo pairs in the Unity game engine", 2008 \\ \href{http://paulbourke.net/stereographics/Unitystereo/}{http://paulbourke.net/stereographics/Unitystereo/}


\bibitem{sub2r}
SUB2R - High speed cameras for computer vision \\ \href{http://www.sub2r.com}{http://www.sub2r.com}

\bibitem{ptgrey}
PTGREY - High speed cameras for computer vision \\ \href{http://www.ptgrey.com}{http://www.ptgrey.com}

\bibitem{ar_rift}
W. Steptoe, S. Julier, A. Steed,
``Presence and Discernability in Conventional and Non-Photorealistic Immersive Augmented Reality",
Mixed and Augmented Reality (ISMAR), 2014 IEEE International Symposium on. IEEE, 2015,
\doi{10.1109/ISMAR.2014.6948430}




\bibitem{fisheye_lens}
D. Brooks,
`` Lenses and lens accessories: a photographer's guide", 1982, p. 29. \\ISBN: 9780930764340.

\bibitem{book_stereographic_projection}
R. J. Lisle, P. R. Leyshon,
``Stereographic Projection Techniques for Geologists and Civil Engineers",
Cambridge University Press, 2004
\doi{10.1017/CBO9781139171366}

\bibitem{immersive_displays}
E. Lantz, 
``A survey of large-scale immersive displays", Proceedings of the 2007 workshop on Emerging displays technologies: images and beyond: the future of displays and interaction, ACM International Conference Proceeding Series; Vol. 252, 2007,
\doi{10.1145/1278240.1278241}

\bibitem{omni_fisheye}
P. Burke,
``Omni-directional Stereoscopic Fisheye Images for Immersive Hemispherical Dome Environments",
WASP, University of Western Australia, Computer Games and Allied Technology,136-143, 2009

\bibitem{cg_projections}
D. Salamon, 
``Transformations and Projections in Computer Graphics", Springer London, pp 145-220, 2005,
\doi{10.1007/978-1-84628-620-9}


\bibitem{link_dissecting_camera_matrix}
K. Simek, ``Dissecting the camera matrix", June 2013\\ \href{http://ksimek.github.io/2013/06/03/calibrated\_cameras\_in\_opengl/}{http://ksimek.github.io/2013/06/03/calibrated\_cameras\_in\_opengl/}


\bibitem{cg_rendering}
N. Kurachi, 
``The magic of computer graphics", CRC Press, cited from Chapter 5: "image based rendering", 2011,\\
Print ISBN: 978-1-56881-577-0\\
eBook ISBN: 978-1-4398-7357-1

\bibitem{link_vrui}
VRUI - Vrui Toolkit Official Homepage \\ \href{http://idav.ucdavis.edu/~okreylos/ResDev/Vrui/}{http://idav.ucdavis.edu/~okreylos/ResDev/Vrui/}

\bibitem{link_fisheye_undistortion}
Stackoverflow - ``Correcting Fisheye Distortion Programmatically", an example result of fish-eye undistortion and its problems \\ \href{http://stackoverflow.com/questions/2477774/correcting-fisheye-distortion-programmatically}{http://stackoverflow.com/questions/2477774/correcting-fisheye-distortion-programmatically}

\bibitem{precise_fisheye_calib}
M. Kedzierski, P. Walczykowski, R. Kaczynski,
``Precise calibration of fisheye lens camera system and projection model", 2006

\bibitem{link_calib3d_opencv}
OpenCV - Calib3d implementation in OpenCV for fisheye model \\ \href{http://docs.opencv.org/master/db/d58/group_\_calib3d_\_fisheye.html}{http://docs.opencv.org/master/db/d58/group\_\_calib3d\_\_fisheye.html}


\bibitem{link_matlab_ocamcalib}
OcamCalib - Camera calibration tool for Matlab \\ \href{https://sites.google.com/site/scarabotix/ocamcalib-toolbox}{https://sites.google.com/site/scarabotix/ocamcalib-toolbox}

\bibitem{link_aruco_ogre}
arUco - Example with Ogre3D \\ \href{https://www.youtube.com/watch?v=CzD48UkGsK8}{https://www.youtube.com/watch?v=CzD48UkGsK8}

\bibitem{link_optimized_undistort_opencv}
openCV - How to improve OpenCV performance on lens undistortion from a video feed \\ \href{http://blog.nishihara.me/opencv/2015/09/03/how-to-improve-opencv-performance-on-lens-undistortion-from-a-video-feed/}{http://blog.nishihara.me/opencv/2015/09/03/how-to-improve-opencv-performance-on-lens-undistortion-from-a-video-feed/}


\iffalse
\fi


\end{thebibliography}



\end{document}
